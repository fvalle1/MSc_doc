%%%%%%%%%%%%%%%%%%%%%%%%%%%%%%%%%%%%%%%%%%%%%%%%%
%
%	MSc THESIS TEMPLATE
%	developed for my master thesis at the Universitá di Torino
%
%	by Eugenio Senes (eugenio.senes@gmail.com)
%
%	released under MIT license, so share, modify and enjoy, but quoting the author !
%
%%%%%%%%%%%%%%%%%%%%%%%%%%%%%%%%%%%%%%%%%%%%%%%%%

%% DOCUMENT CLASS (alternative to book is 'report')
% Print just right page or both sides (comment the other one)
%\documentclass[12pt,a4paper,openright,oneside]{book}	%%One sided
%\documentclass[12pt,a4paper,openright,twoside]{book}	%%Double sided
\documentclass[12pt,a4paper,oneside]{book}

%% SET MARGINS OF THE PAGES
\usepackage[a4paper,portrait, left=35mm, right=20mm, top=35mm, bottom=30mm]{geometry}

%% HEADERS AND FOOTERS
\usepackage{fancyhdr}
\pagestyle{fancy}
\fancyhf{} 			%clears default header and footer
\fancyhead[RO]{\leftmark}
\fancyhead[LE]{\leftmark}
\fancyfoot[RO]{\thepage}
\fancyfoot[LE]{\thepage}
%\rhead{} 			%right head
%\lhead{} 	%left head
%\rfoot{\thepage}
%%consider using also chead, cfoot, lfoot
%coherce the plain stile to this (e.g. the first page of every chapter)
\fancypagestyle{plain}{
	\fancyhf{}
	\rfoot{\thepage}
	\renewcommand{\headrulewidth}{0pt}
	\renewcommand{\footrulewidth}{0pt}
}

%% CLEAR PAGE WITHOUT NUMBER AT THE BEGINNING OF CHAPTERS
\let\origdoublepage\cleardoublepage
\newcommand{\clearemptydoublepage}{%
  \clearpage
  {\pagestyle{empty}\origdoublepage}%
}
%% ALLOW PAGE ROTATION
\usepackage{lscape}

%% HYPERTEXT SETUP
\usepackage[unicode=true]{hyperref}
\hypersetup{
	pdfauthor={Filippo Valle},
	pdftitle={MSc thesis},
	pdfsubject={Topic Modelling},
	pdfkeywords={network theory, graph theory, topic modelling, machine learning, clustering},
	breaklinks=true,
	colorlinks=true,
	citecolor=blue,
	urlcolor=blue,
	linkcolor=black
}

\author{Filippo Valle}
\date{\today}
\title{MSc thesis}

%% FONTS AND SYMBOLS
\usepackage[utf8]{inputenc}	%%input font setting
\usepackage{ textcomp }
\usepackage[T1]{fontenc} 		%%font for automatic recognition of letters with the accent
\usepackage{amsfonts}		%%fonts for the mathematical rendering of formulas
\usepackage{amssymb}
\usepackage{amsmath}
%% CHAPTERS STRUCTURE
\usepackage[english]{babel} %%Set English as main language of the document
%% FIGURES
\usepackage{graphicx}
%% TABLES
\usepackage{booktabs}		%%allow use of \toprule, \midrule, \bottomrule in tables
%%CAPTIONS
\usepackage{caption}
%% CODE LISTINGS
\usepackage{listings}		%%allow to use code listings
\usepackage{rotating} %%sideways/
\usepackage{placeins} %%\FloatBarrier

\lstdefinestyle{myPython}{
backgroundcolor=\color{white},   % choose the background color; you must add \usepackage{color} or \usepackage{xcolor}; should come as last argument
basicstyle=\footnotesize\ttfamily,        % the size of the fonts that are used for the code
breakatwhitespace=false,         % sets if automatic breaks should only happen at whitespace
breaklines=true,                 % sets automatic line breaking
captionpos=b,                    % sets the caption-position to bottom
commentstyle=\color{mygray},    % comment style
keepspaces=true,                 % keeps spaces in text, useful for keeping indentation of code (possibly needs columns=flexible)
keywordstyle=\color{mygreen},       % keyword style
identifierstyle=\color{black},
language=python,                 % the language of the code
morekeywords={*, as},
numbers=left,                    % where to put the line-numbers; possible values are (none, left, right)
numbersep=5pt,                   % how far the line-numbers are from the code
numberstyle=\tiny\color{mygray}, % the style that is used for the line-numbers
rulecolor=\color{black},         % if not set, the frame-color may be changed on line-breaks within not-black text (e.g. comments (green here))
showspaces=false,                % show spaces everywhere adding particular underscores; it overrides 'showstringspaces'
showstringspaces=false,          % underline spaces within strings only
showtabs=false,                  % show tabs within strings adding particular underscores
stepnumber=2,                    % the step between two line-numbers. If it's 1, each line will be numbered
stringstyle=\color{myblue},     % string literal style
tabsize=2,	                   % sets default tabsize to 2 spaces
}

\usepackage{float} %forza [htb]
\usepackage{color}

\definecolor{mygreen}{rgb}{0,0.6,0}
\definecolor{mygray}{rgb}{0.5,0.5,0.5}
\definecolor{mymauve}{rgb}{0.58,0,0.82}
\definecolor{myblue}{rgb}{0.44,0.62,0.78}
\definecolor{pythonred}{RGB}{255,0,0}
\definecolor{pythongreen}{RGB}{0,150,57}
\definecolor{pythonblue}{RGB}{58,0,242}
\definecolor{pythonorange}{RGB}{255,122,0}
\definecolor{pythonyellow}{RGB}{250,255,0}

%% HYPENATON
\hyphenation{te-si}	%manual hyphenation

%%commands
\newcommand{\avg}[1]{\left\langle #1\right\rangle}
\newcommand{\var}[1]{\sigma^2_{#1}}
\newcommand{\draft}[1]{\textcolor{red}{#1}}

%%%%%%%%%%%%%%%%%%%%%%%%%%%%%%%%%%%%%%%%%%%%%%%%%
%%%% BEGIN DOCUMENT
\begin{document}
%%%%%% HEAD  OF THE DOCUMENT
\frontmatter
%%FRONT PAGE
%\input{head/frontPage.tex}
\begin{titlepage}
%upper part
\begin{center}
{{\Large{\textsc{Universit\`a degli studi di Torino \\}}}} \vspace{5mm} {\small{\bf SCUOLA DI SCIENZE DELLA NATURA\\ \vspace{3mm}
Corso di Laurea Magistrale in Fisica dei Sistemi Complessi}}
\vspace{5mm}
\end{center}
%logo
\begin{center}
\includegraphics[scale=.3]{head/logo.png}
\end{center}
%title
\begin{center}
\vspace{5mm}
{\large{\bf Tesi di Laurea Magistrale\\}}
\vspace{5mm}
{\LARGE{\bf THE FANCY TITLE\\ OF MY FANCY THESIS\\}}
%\vspace{5mm}
%{\LARGE{\bf SECOND ROW TITLE}}
\end{center}
\vspace{20mm}
%reatori e candidato
\vspace{11mm}
\par
\noindent
\begin{minipage}[t]{0.47\textwidth}
{\large{\bf Relatore:\\
Prof. Michele Caselle}}\\
\vspace{4mm}
\\
{\large{\bf Co-relatore:\\
Dott. Matteo Osella }}
\vspace{8mm}
{\large{\bf \\ Controrelatore:\\
Dott. Matteo Cereda}}
\end{minipage}
\hfill
\begin{minipage}[t]{0.47\textwidth}\raggedleft
\vspace{16mm}
{\large{\bf Candidato:\\
Filippo Valle}}
\end{minipage}
\vspace{9mm}
\begin{center}
{\large{\bf 
Anno Accademico 2018/2019}}
\end{center}

\end{titlepage}
\clearemptydoublepage
%%DEDICATION (the initial quote)
\thispagestyle{empty}
\begin{flushright}

\vspace*{60mm}

The imagination of nature is far, far greater \\
than the imagination of man\footnote{What do you care what other people think?}.
\\
\vspace{4mm}
Richard P. Feynman.\textit{}\\

\end{flushright}
\clearemptydoublepage
%%ABSTRACT
\chapter*{Abstract}
The interest in studying complex systems is increasingly spreading.
Complex systems can be found anywhere and many common behaviours are
observable, systems with different origins and purposes may share, for
instance, some statistical laws.

An example can be the Zipf's law, well-known in linguistics and texts
analysis. It can be easily observed in the distribution of gene expressions
in different samples of cancer tissues.

In recent years datasets with a large amount of cancer samples' data are
available, the most complete is The Cancer Genome Atlas (TCGA). From
this dataset, it is easy to get, for example, gene expression data from
RNA-sequencing experiments together with a lot of information about the
samples themselves.

If one studies the number of samples in which a gene is expressed above
a certain threshold, the so-called occurrence, it is easily verified
that there are different kinds of genes. Some are present in the
majority of samples, some others are present only in a subset of the
whole dataset. The same behaviour can be found analysing words in
a corpus of texts; some words, such as \emph{the}, are present everywhere,
other specific words are present only in texts regarding a certain
subject. This suggests that there are similarities between a system of
words and documents and a system of genes and samples.

Given a corpus of documents, they can be classified by their specific
subject. Similarly, a set of samples can be classified, for
example, by the tissue it comes from or by the type of the disease it is
referred to.

The similarities between gene expression data and linguistics suggest
the possibility to use topic modelling to classify data and separate
samples and genes in different clusters. Topic modelling is a set of
clustering algorithms in networks' theory. Given a set of words and
documents, it describes documents as a mixture of topics. Topics are
nothing but communities of words each one with a given probability.

Purpose of this work is to build a bipartite network of genes and
samples and use topic modelling to find communities. The goal is to
separate samples depending on the site the tumour was and the disease
type of the sample. Moreover, it is possible to separate genes depending
on their specific functions. Once a community structure of genes
emerges, it is possible to run a hypergeometric test on the whole set to verify if they reveal some type of enrichment and to inspect
their common properties.

The specific algorithm used in this work is particularly unique because
it needs no priors and makes no assumptions on the data; moreover, it can
be set to accept overlapping clusters so it is possible to find genes
belonging to different topics and can be hierarchic all these facts
empower a lot of new possibilities to investigate the network.

A hierarchic approach make it possible to classify data at different
layers. An ideal goal would be to separate healthy and diseased samples
at the first layer, then separate by tissue, then by tumour type and so
on.

\clearemptydoublepage
%%INDEXES 
%summary
\tableofcontents
\clearemptydoublepage

%%%%%% BODY OF THE DOCUMENT
\mainmatter
%%INTRODUCTION
\chapter*{Introduction}\label{ch:intro}
\addcontentsline{toc}{chapter}{Introduction}
In recent years the study of complex systems is becoming more interesting especially when some different systems that share  some fundamental properties are found. Network theory has been proven to be a useful proxy to model and represent such complex systems.

This work wants to study and find universal statistical laws in different kinds of biological systems. If one finds that two different systems share some statistical laws and that they have a somehow similar data structure, therefore it is possible to use tools developed from different fields to study and gain more information. In particular two datasets containing information about some human healthy and diseased tissues will be analysed. These data come from biological experiments of RNA-sequencing.

The ultimate goal of this work would be to study, develop and build a machine learning's model which is able to classify cancer tissues and gain information from healthy tissues as well. Separate cancer types and ultimately sub-types is not always easy clinically and that's why the interest in developing a method able to classify well this kind of data is increasing~\cite{Farver2018}.

The methods to gain this goal are derived firstly from linguistics; in particular, a topic model approach will be widely described. A hierarchic approach will be useful to gain different layers of information.

In chapter~\ref{ch:data}, I will describe the datasets used and introduce some basic biological properties of these datasets. In particular, I'll use two datasets of gene expression data from cancer and healthy tissues.

In chapter~\ref{ch:structure}, I will describe the basics of component systems and give some useful mathematical definitions. Here it will be shown that RNA-sequencing data have many aspects in common with linguistics data. Examples of Zipf's law, well-known and in-depth studied in linguistics, will demonstrate that different sources of data (genomic and linguistics) can share some statistical properties. Some analyses will be shown to explain the different behaviour of different tissues.

In chapter~\ref{ch:scalinglaws}, I will study the gene expression across samples of all the genes. This analysis is preparatory to the following sections where some gene selection would be necessary.

Demonstrated that linguistics and biological data share some statistical laws, in chapter~\ref{ch:topicmodelling}, the main one, I will describe how topic modelling can perform network analysis on these datasets. Topic modelling is an advanced clustering algorithm developed in linguistics to classify text and used in different fields of science. Different approaches to topic modelling are possible starting from the standard ones~\cite{Zhou2016} to some new proposals~\cite{Lancichinetti2015,Martini2017,gerlach2018network}. Using topic modelling one would find the inner structure of the data. One would find clusters such that all samples in a cluster share the tissue or the tumour type. Benchmarks and metrics to test and evaluate this algorithm will be widely discussed.

In chapter~\ref{ch:conclusions}, I will sum up the results and propose some future developments of this work.

Many methods of the pipeline, written in C\texttt{++} using openMP and Boost~\cite{siek2002boost}, are encapsulated in a tool available at~\url{https://github.com/fvalle1/tacos}. During this work, I used different python libraries such as pandas~\cite{mckinney2010data}, scipy~\cite{jones2014scipy}, numpy~\cite{oliphant2006guide} and matplotlib~\cite{hunter2007matplotlib}. Some advanced analysis required Tensorflow~\cite{tensorflow2015-whitepaper} and pySpark~\cite{Zaharia:2016:ASU:3013530.2934664}. The topic modelling stochastic block model's minimization functions are implemented in the graph-tool library~\cite{peixoto_graph-tool_2014}. Computing resources were made available by EGI Foundation~\cite{fernandez2015egi} and from C$^{\text{3}}$S~\cite{occamchep}.

The full work repository is available on GitHub\textsuperscript{\tiny\textcopyright} at~\url{https://github.com/fvalle1/master_thesis} and runnable as a Docker\textsuperscript{\tiny\textcopyright} container that can be pulled from~\url{https://hub.docker.com/r/fvalle01/thesis}. 
\clearemptydoublepage
%% CHAPTERS
% add any further chapter file here
\chapter{Data presentation}\label{ch:data}
\section{RNA-sequencing}\label{sec:rnaseq}
Data considered in this work come from RNA-sequencing~\cite{wang2009rna} experiments. These experiments aim to quantify how much a gene is expressed in a particular sample of a tissue. RNA-Sequencing data provide a unique snapshot of the transcriptomic status of the sample. 

Briefly, long RNAs are first converted into a library of complementary DNA (cDNA) fragments through either RNA fragmentation or DNA fragmentation. Sequencing adaptors are subsequently added to each cDNA fragment and a short sequence is obtained from each cDNA using high-throughput sequencing technology. The resulting sequence reads are aligned with the reference genome or transcriptome, and classified as three types: exonic reads, junction reads and poly(A) end-reads. These three types are used to generate a base-resolution expression profile for each gene.

The general steps to prepare a cDNA library for sequencing are, in general:
\begin{itemize}
\item RNA Isolation: RNA is isolated from tissue and the amount of genomic DNA is reduced;
\item RNA selection/depletion: to analyse signals of interest, the isolated RNA can either be kept as is or filtered for RNA that binds specific sequences. The non-coding RNA is removed because it represents over 90$\%$ of the RNA in a cell, which, if kept, would drown out other data in the transcriptome;
\item cDNA synthesis: RNA is reverse transcribed to cDNA (DNA sequencing technology is more mature). Fragmentation and size selection are performed to purify sequences that are the appropriate length for the sequencing machine.  Fragmentation is followed by size selection when either small sequences are removed or a tight range of sequence lengths are selected. Because small RNAs like miRNAs are lost, these are analysed independently. The cDNA for each experiment can be indexed with a hexamer or octamer barcode, so that these experiments can be pooled into a single lane for multiplexed sequencing.
\end{itemize}
To collect gene expression data is sufficient to count how many reads are mapped to a specific exon or gene. The ultimate output of this analysis, where this work begins, are nothing but lists of gene expression values for each sample.
\paragraph{Normalization}\mbox{}\\
Usually gene expression data can be normalized in different ways, for example, it is possible to use:
\begin{itemize}
	\item counts,
	\item RPK,
	\item TPM,
	\item FPKM.
\end{itemize}
Count reads correspond to raw data. Counts need no normalization to be treated but may be biased. For example, longer genes may have more reads than shorter ones just because they are longer. That's why other kind of normalization can be used. Note that this is not always the case: in fact, some sequencing techniques consider just the start of the gene, so the gene length doesn't matter.   

RPK\footnote{Reads Per Kilobase of transcript} normalization removes the length bias by dividing counts by the gene length $L$: \(\text{RPK}=\frac{\text{counts}}{L}\). This solves some problems but doesn't take care of the different sizes of the transcript in different samples.

FPKM\footnote{Fragments Per Kilobase of transcript per Million mapped reads} calculation normalizes read count by dividing it by the gene length and by the total number of reads mapped to protein-coding genes in that sample.
\[
\text{FPKM} = \frac{RC_g*10^9}{RC_{pc}*L}
\]
being
\begin{itemize}
	\item $RC_g$: Number of reads mapped to the gene
	\item $RC_{pc}$: Number of reads mapped to all protein-coding genes
	\item $L$: Length of the gene in base pairs.
\end{itemize}
When dealing with FPKM some thresholds are put in particular FPKM below $0,1$ and above $10^5$ should not be considered, maybe these values come from some kind of experimental error.

TPM\footnote{Transcript Per Million} tries to unify the sizes of the samples: $\text{TPM} = \frac{RC_g*10^9}{\sum_{g\prime} \left(\frac{RC_{g^\prime}}{l_{g^\prime}}\right) RC_{pc}*L}$.

In this work the idea is not to introduce any normalization, so when possible raw counts will be considered. Sometimes especially if it is necessary to compare different sources TPM or FPKM will be taken in account. Some analysis as the sizes' distribution needs not to be made with TPM, because the quantity that's going to be studied is the quantity the normalization trashes out.

\section{Datasets}\mbox{}\\
In this work two datasets were used. The first one contains RNA-sequencing data of post-mortem collected samples. It is the Gene Tissue Expression (GTEx) dataset~\cite{carithers2015novel}. GTEx \textit{2016-01-15 v7 RNASeQCv1.1.8} version was downloaded\footnote{\url{https://gtexportal.org/home/datasets}}.
GTEx contains $11688$ samples of $53$ tissues. For many of them, a sub-tissue label is available. As highlighted in~\cite{dey2017visualizing} these data present a challenge to clustering tools, because of both the relatively large number of samples and the complex structure created by the inclusion of many different tissues.

The other dataset considered is The Cancer Genome Atlas (TCGA) dataset~\cite{grossman2016toward}. Data were collected via Genomic Data Commons tools\footnote{\url{https://portal.gdc.cancer.gov}} considering \textit{Gene Expression Quantification} as data type, \textit{Transcriptome Profiling} as data category, \textit{RNA-Seq} as experimental strategy, \textit{HTSeq - Counts} or \textit{HTSeq - FPKM} as workflow type. On TCGA there are $12683$ samples and $68$ primary site or tissues. On this dataset there is a great quantity of metadata, in particular the \textit{disease type} will be considered in this work.

Another source of data used in this work is~\cite{Wang2017} which authors tried to unify GTEx and TCGA~\cite{Betel2018}, when possible.

\paragraph{Protein-coding}
Each dataset contains infos on approximately $60000$ elements with a different \textit{ENSG} identifier. Only $\simeq 20000$ of this are protein-coding genes, using Ensemble\footnote{\url{https://ensemble.org}} protein-coding genes are selected. In chapters~\ref{ch:structure} and~\ref{ch:scalinglaws} some analysis explaining the different behaviour of coding and non-coding genes will be described.

\chapter{Data structure}\label{ch:structure}

The data studied in this work can be represented as component systems. These component systems can be represented by a two dimensional matrix in which rows represent components and columns are the possible realizations buildable given subset of the components. The entries of this matrix are the number of the components on the row needed during the realization of the column. In figure~\ref{fig:componetstable} an example of this kind of matrices.

%%data definitions
\section{Component systems}
The most common example of such systems is a set of books. In this case one puts on the rows the words in the whole vocabulary and the books' titles on the columns. The entry that corresponds to row $i$ and column $j$ is the number of times the word $i$ appears in the book $j$. The same happens if one considers Wikipedia's pages. Other examples are: Lego$\textsuperscript{\tiny\textregistered}$ sets where components are the Lego$\textsuperscript{\tiny\textregistered}$ bricks and realizations the Lego$\textsuperscript{\tiny\textregistered}$ packages, and protein domains; all these were described and well studied in~\cite{mazzolini2018heaps, Mazzolini2018zipf}.

Given a matrix with $N$ components on the rows and $R$ realizations on the columns and relative abundances $n_{ij}$ as the entries, it is interesting to study some quantities that are universal and general characteristics of component systems.

First of all, the \textbf{occurrence} of a component is defined as 
\begin{equation}\label{eq:occurrence}
O_i=\frac{\sum_{j=1}^{R}(1-\delta_{n_{ij},0})}{R}.
\end{equation}
It is the fraction of realizations in which the component's abundance is not null. A component that is present in all the realizations has got $O_i=1$, the ensemble of the components with $O_i=1$ is known as the \textbf{core}. Components with high ($\simeq 1$) occurrence are present in mostly all realizations of the datasets. In linguistics articles, such as \textit{the}, are present everywhere, so they have high occurrence. Components with low occurrence ($\simeq 0$) are present only in a few realizations and are the most specific ones~\cite{altmann2016statistical}.

The sum across all columns, or the number of times component $i$ appear in the dataset, is called \textbf{abundance} of the component and is defined as
\begin{equation}\label{eq:abundance}
a_i=\sum_{j=1}^{R}n_{ij};
\end{equation}
dividing this by the global abundance, or the total number of components in the dataset,
\begin{equation}
  a=\sum_{i=1}^{N}a_i
\end{equation}
naturally brings to the \textbf{frequency of a component} in the whole corpus
\begin{equation}\label{eq:fi}
f_i=\frac{a_i}{\sum_{c=1}^{N}a_{c}}.
\end{equation}
The abundance of a component divided by the sum of all the abundances in a realization gives the \textbf{frequency} of the component in that specific realization
\begin{equation}
f_{ij}=\frac{n_{ij}}{\sum_{c=1}^{N}n_{cj}}.
\end{equation}

The sum of all abundances in a realization,
\begin{equation}\label{eq:size}
M_j=\sum_{c=1}^{N}n_{cj}
\end{equation}
represents the \textbf{size} of the realization. In gene expression this is the size of the transcript.

It is expected that frequencies distribute according to the so-called Zipf's law
\begin{equation}\label{eq:zipf}
f_i\propto r_i^{-\alpha}
\end{equation}
where $r$ is the rank: the position of a component when sorting, in descending order, all components by their frequencies in the whole dataset.

\section{Universal laws in RNA-Seq}
\subsection{TCGA}

\subsection{GTEx}


%%null model
\section{Null model construction}\label{sec:nullmodel}
The kind of data considered in this work comes from RNA Sequencing experiments. This experiments use wet biology methods to extract information from samples. If one imagines it exists an unknown function that describes the gene expression across the samples considered, what experimenters people do is to sample  this function, picking up some genes.

In this section it is described a null model of sampling, this is useful to verify if the data distributions seen are just an effect of this experimental sample or if they carry some useful and interesting information.

As described in~\cite{mazzolini2018heaps} a random matrix has to be created. This matrix is a collection of components and realizations exactly as~\ref{fig:componetstable}. The values of abundances of each component in each realization $n_{i j}$ are randomly assigned with a probability determined by 
the global abundance in the whole dataset~\ref{eq:abundance}. Values of each column are extracted until the size~\ref{eq:size} is 
reached. Strictly speaking it is a multinomial process
\begin{equation}
P\left({n_i};M\right)=\frac{M!}{\prod_{i=1}^{N} n_i}\prod_{i=1}^N f_i^{n_i}
\end{equation}
where $n_i$ is the number of components with frequency $f_i$, being $f_i=\frac{a_i}{\sum_{i=1}^{N}a_{i}}$ as defined in~\ref{eq:fi}.

Figure~\ref{fig:structure/randomsampling} shows an example of this, $M$ components are picked up with respect to their frequency in the dataset. The most abundant components, which are also the ones with higher frequency (frequency is nothing but the normalised abundance), have a greater probability to be picked up.
\begin{figure}[htb!]
    \centering
    \includegraphics[width=0.8\linewidth]{pictures/structure/randomsampling.png}
    \caption{Random sampling of components to build a realization of size $M$}
    \label{fig:structure/randomsampling}
\end{figure}

Using this construction on data of counts on both dataset, by definition the Zipf's law sampled are identical to the data's one.
\begin{figure}[htb!]
\begin{minipage}{0.5\textwidth}
    \centering
    \includegraphics[width=0.95\linewidth]{pictures/structure/tcga/globalzipf_null.pdf}
\end{minipage}
\hspace{2mm}
\begin{minipage}{0.5\textwidth}
    \centering
    \includegraphics[width=0.95\linewidth]{pictures/structure/gtex/globalzipf_null.pdf}
\end{minipage}
\caption{Zipf's law sampled; TCGA(left) and GTEx (right)}
\label{fig:structure/globalzipf_null}
\end{figure}
By construction the distribution of the sizes of the sampling and of the data are identical.
\begin{figure}[htb!]
\begin{minipage}{0.5\textwidth}
    \centering
    \includegraphics[width=0.95\linewidth]{pictures/structure/tcga/sizeDistr_null.pdf}
\end{minipage}
\hspace{2mm}
\begin{minipage}{0.5\textwidth}
    \centering
    \includegraphics[width=0.95\linewidth]{pictures/structure/gtex/sizeDistr_null.pdf}
    \end{minipage}
\caption{Distribution of sizes $M$; TCGA(left) and GTEx (right)}
    \label{fig:structure/sizeDistr_null}
\end{figure}

Looking at the $U$s, it is evident that data is different from sampling. This is a signal that the null model is not enough to explain the data matrices. In particular from figure~\ref{fig:structure/globalU_null} it is evident that the null model generate the matrices in a manner such that more components have high occurrence with respect to the original data. This can be easily explained, in fact in real world there are some genes that are highly expressed but only in a subset of the whole dataset; these genes are specific for certain type of samples. The null model gets the information the such genes are highly expressed from the abundance and so samples these quite often (components with high abundance have a greater chance to be picked up by the null model sampling).
\begin{figure}[htb!]
\begin{minipage}{0.5\textwidth}
    \centering
    \includegraphics[width=0.95\linewidth]{pictures/structure/tcga/globalU_null.pdf}
\end{minipage}
\hspace{2mm}
\begin{minipage}{0.5\textwidth}
    \centering
    \includegraphics[width=0.95\linewidth]{pictures/structure/gtex/globalU_null.pdf}
    \end{minipage}
\caption{Occurrence distributions; TCGA(left) and GTEx (right)}
\label{fig:structure/globalU_null}
\end{figure}

Looking at the Heaps's law~\cite{Heaps:1978:IRC:539986} 
, again the curves differ and the null model is not enough complete to explain the trend. In figure~\ref{fig:structure/heaps_null} the Heaps's law is presented compared to the one obtained by sampling, note that each data point share the abscissa with a sampling one (figures~\ref{fig:structure/sizeDistr_null} are nothing but the histograms of the abscissas of~\ref{fig:structure/heaps_null}). It happens that the sampling curve is above the data's one. This means that to build a sample of size $M$ just by sampling it is necessary to use a greater number of different genes than the number of different genes actually expressed in nature. In other words in real world are expressed only the genes that are really useful in the sample, and this is not describable just by sampling. This fact is coherent with the fact that the $U$s differ.
\begin{figure}[htb!]
\begin{minipage}{0.5\textwidth}
    \centering
    \includegraphics[width=0.95\linewidth]{pictures/structure/tcga/heaps_null.pdf}
    \end{minipage}
\hspace{2mm}
\begin{minipage}{0.5\textwidth}
    \centering
    \includegraphics[width=0.95\linewidth]{pictures/structure/gtex/heaps_null.pdf}
    \end{minipage}
\caption{Heaps' law; TCGA(left) and GTEx (right)}
\label{fig:structure/heaps_null}
\end{figure}
Another way to see this is looking at the histograms of the number of different genes expressed, actually the distribution of the~\ref{fig:structure/heaps_null} y axis. Figure~\ref{fig:structure/diffwordsDistr_null} shows that these distributions are completely different if one looks at the data and at the samples.
\begin{figure}[htb!]
\begin{minipage}{0.5\textwidth}
    \centering
    \includegraphics[width=0.95\linewidth]{pictures/structure/tcga/diffwordsDistr_null.pdf}
    \end{minipage}
\hspace{2mm}
\begin{minipage}{0.5\textwidth}
    \centering
    \includegraphics[width=0.95\linewidth]{pictures/structure/gtex/diffwordsDistr_null.pdf}
    \end{minipage}
\caption{Occurrence distributions; TCGA(left) and GTEx (right)}
\label{fig:structure/diffwordsDistr_null}
\end{figure}

%%tissue separation
\section{Statistical laws differentiate by tissue}
Observing the GTEx dataset of healthy samples it is possible to study how it is possible to see the tissue differentiation and how to study tissues' differences,~\cite{mele2014} suggests the approach.

First of all could be interesting to study which is the fraction of transcript that can be explained by a certain number of genes.
One can reduce the realisations to the ones that share the tissue. Than one estimates the average per each component (gene), at this point one has the average abundance of each gene in a tissue, dividing by the sum of all the components it is possible to obtain the fraction of the total counts in the tissue due to each gene. Sorting from greater to smaller and integrating (cumulative summing) one have the fraction of transcript due to $1, 2, 3\dots$ genes. This is plot in~\ref{fig:structure/gtex/fraction_of_trascriptome}. 
\begin{figure}[htb!]
  \centering
  \includegraphics[width=0.9\linewidth]{pictures/structure/gtex/fraction_of_trascriptome.pdf}
  \caption{The integral of the sorted abundances for each tissue}
  \label{fig:structure/gtex/fraction_of_trascriptome}
\end{figure}
Here, if a curve is steep it means that a few genes' counts represent a great fraction of the total. If a curve is smooth it means that many genes are necessary to describe the whole trascriptome for that particular tissue.
This analisys shows that different tissues have a different complexity in terms of the number of genes necessary to build the trascriptome (in average).
In figure~\ref{fig:structure/gtex/fraction_of_trascriptome_Brain} the same analisys is done for the sub-tissues of Brain, also this sub-type separate by tissue.
\begin{figure}[htb!]
  \centering
  \includegraphics[width=0.9\linewidth]{pictures/structure/gtex/fraction_of_trascriptome_Brain.pdf}
  \caption{The integral of the sorted abundances for sub-types of Brain. This is done using TPM to avoid biases due to gene lengths. Blood is plotted for reference.}
  \label{fig:structure/gtex/fraction_of_trascriptome_Brain}
\end{figure}

Coming back to the Zipf's law~\ref{eq:zipf}, it is now obvious that~\ref{fig:structure/gtex/fraction_of_trascriptome} represents nothing but the integral of the Zipf's law. So estimating the Zipf looking at a tissue a time, it is evident that each tissue has its particular slope. The steeper the Zipf the simplest is the tissue: the transcript can be described with a few genes. In figure~\ref{fig:structure/gtex/zipf_tissue} the tissue with an extreme behaviour.
\begin{figure}[htb!]
  \centering
  \includegraphics[width=0.6\linewidth]{pictures/structure/gtex/zipf_tissue.pdf}
  \caption{The integral of the sorted abundances for each tissue}
  \label{fig:structure/gtex/zipf_tissue}
\end{figure}

The point where the~\ref{fig:structure/gtex/fraction_of_trascriptome} reaches $1$ corresponds to the total number of genes expressed, the remaining ones have a $0$ expression and do not contribute to the transcript. This can be visualised again with the Heaps' law. In figure~\ref{fig:structure/gtex/heaps_tissue} it is evident that there is some kind of tissue differentiation even when looking at the Heaps' law.
\begin{figure}[htb!]
  \centering
  \includegraphics[width=0.6\linewidth]{pictures/structure/gtex/heaps_tissue.pdf}
  \caption{The integral of the sorted abundances for each tissue}
  \label{fig:structure/gtex/heaps_tissue}
\end{figure}

\begin{figure}[htb!]
  \centering
  \includegraphics[width=0.6\linewidth]{pictures/structure/gtex/heaps_tissue_disease.pdf}
  \caption{The integral of the sorted abundances for each tissue}
  \label{fig:structure/gtex/heaps_tissue_disease}
\end{figure}

All these analysis suggest that there must be a sort of hidden structure in the data that is somehow related with the tissue each sample comes from. In particular there are many different Zipf's laws hidden behind the data and each sample is build looking at one of these a time. Also given two samples with a similar size, it happens that the number of genes necessary to build that realisation is not always the same (shown by Heaps' law) and it is somehow related to the tissue of the sample.


\chapter{Scaling laws}\label{ch:scalinglaws}
One of the goals of this work is to search, reveal, study and use universal laws in bulk gene expression data~\nocite{altmann2016statistical}.
As was done in chapter~\ref{ch:structure} approaches from different field of science are considered at this point.

In can be interesting to study the behaviour of the gene expression across samples.

Given a matrix of components and realisations as~\ref{fig:componetstable} with expression entries $n_{i j}$ it is possible to estimate the mean of a row $m_i=\avg{n_{i
 j}}_j$ and its variance $\var{i}=\avg{n_{i j}^2}_j - \avg{n_{i j}}^2_j$.

\paragraph{Variance versus mean}\mbox{}\\
First of all, it could be interesting to study the variance of expression $\var{\mathrm{counts}}$ versus 
the average $\avg{\mathrm{counts}}$ across tissues.
%%all genes
\begin{figure}[htb!]
    \centering
    \includegraphics[width=0.9\linewidth]{pictures/scalinglaws/gtex/allgenes/varmean_loglog.png}
    \caption{Variance versus average. In \textcolor{pythonred}{red} the Poisson-like scaling, in \textcolor{pythongreen}{green} the Taylor-like scaling. All genes are considered.}
    \label{fig:scalinglaws/gtex/allgenes/varmean_loglog_density}
\end{figure}
In figure~\ref{fig:scalinglaws/gtex/allgenes/varmean_loglog_density} the scatter plot of variance versus mean reveals some interesting facts.
First of all, it is evident that data have a double scaling behaviour: when the mean is small ($\lesssim 1$) data have a Poisson-like scaling ($\var{\mathrm{counts}} \sim \avg{\mathrm{counts}}$), at higher means data have instead a quadratic scaling ($\var{\mathrm{counts}} \sim \avg{\mathrm{counts}}^2$) known in ecology as Taylor's law~\cite{Eisler2008}. This means that at low averages data's behaviour is due to the sampling process; on the contrary, Taylor's law reveals the non-trivial distribution across samples of the gene expression.
Another interesting fact is that looking at the density of points (colours in figure~\ref{fig:scalinglaws/gtex/allgenes/varmean_loglog_density}) two clouds of points emerge: one at low averages, one at high averages. These correspond to coding and non-coding genes, remembering section~\ref{sec:universallaws} these two kind of genes have different behaviours: coding genes are highly expressed in the majority of the samples, non-coding ones are less expressed (and so less sampled) in a few samples. 

\paragraph{Coefficient of Variation}\mbox{}\\
A similar analysis, common in literature, is the analysis of the coefficient of variation squared $CV^2=\frac{\var{\mathrm{counts}}}{\avg{\mathrm{counts}}^2}$ represented in figure~\ref{fig:scalinglaws/gtex/allgenes/cvmean_loglog}.
\begin{figure}[htb!]
    \centering
    \includegraphics[width=0.9\linewidth]{pictures/scalinglaws/gtex/allgenes/cvmean_loglog.png}
    \caption{Coefficient of variation squared versus average. In \textcolor{pythonred}{red} the Poisson-like scaling, in \textcolor{pythongreen}{green} the Taylor-like scaling. All genes are considered.}
    \label{fig:scalinglaws/gtex/allgenes/cvmean_loglog}
\end{figure}
The behaviour is complementary to the one discussed above; a double scaling, quite common in the literature looking at single-cell RNA sequencing data~\cite{Islam2013}, is present. Even looking at $CV^2$ it is evident the presence of the coding and non-coding clouds of points. The non-coding genes are on the Poisson-like scaling, $\var{\mathrm{counts}} \sim \avg{\mathrm{counts}}$ so $CV^2=\frac{\var{\mathrm{counts}}}{\avg{\mathrm{counts}}^2}\sim\frac{1}{\avg{\mathrm{counts}}}$, otherwise the protein-coding genes are on the Taylor-like curve $CV^2=\frac{\var{\mathrm{counts}}}{\avg{\mathrm{counts}}^2}\sim \text{constant}$.

\paragraph{Protein-coding genes} can be isolated and considered on their own. The same analysis confirms that the cloud of genes' points on the Taylor-like scaling are effective the protein-coding genes.
\begin{figure}[htb!]
    \centering
    \includegraphics[width=0.9\linewidth]{pictures/scalinglaws/gtex/varmean_loglog_density.png}
    \caption{Variance versus average. In \textcolor{pythonred}{red} the Poisson-like scaling, in \textcolor{pythongreen}{green} the Taylor-like scaling. Only protein-coding genes are considered.}
    \label{fig:scalinglaws/gtex/varmean_loglog_density}
\end{figure}

Following the sampling model of~\cite{Mazzolini2018} summed up in section~\ref{sec:nullmodel} the averages and variances can be estimated on null matrices. In figure~\ref{fig:scalinglaws/gtex/varmean_3sigma} the comparison between real genes and sampling data. The sampling has got a double scaling as well; this is quite interesting, it means that the global scaling is due to the Zipf distribution and the sizes' distribution themselves, they are identical in data and sampling by definition.
Moreover, the sampling points draw a lower bound of the data, this encodes the information that the data are more variable (have higher variance) than just sampling, so there must be some biological information hidden that causes this over-variable behaviour.
\begin{figure}[H]
    \centering
    \includegraphics[width=0.8\linewidth]{pictures/scalinglaws/gtex/varmean_3sigma.png}
    \caption{Variance versus average. In \textcolor{pythonred}{red} the Poisson-like scaling, in \textcolor{pythongreen}{green} the Taylor-like scaling. In \textcolor{pythonorange}{orange} the sampling components. Only protein-coding genes are considered.}
    \label{fig:scalinglaws/gtex/varmean_3sigma}
\end{figure}

Again it is possible to analyse the $CV^2$, this time considering only protein-coding genes. Figure~\ref{fig:scalinglaws/gtex/allgenes/cvmean_loglog} confirms that the cloud of points near the Taylor-like scaling is made of protein-coding genes and a double scaling is seen once again.
\begin{figure}[htb!]
    \centering
    \includegraphics[width=0.9\linewidth]{pictures/scalinglaws/gtex/cvmean_loglog_density.png}
    \caption{Coefficient of variation squared versus average. In \textcolor{pythonred}{red} the Poisson-like scaling, in \textcolor{pythongreen}{green} the Taylor-like scaling. Only protein coding genes are considered.}
    \label{fig:scalinglaws/gtex/cvmean_loglog}
\end{figure}

In figure~\ref{fig:scalinglaws/gtex/cvmean_loglog_sampling} the same plot compared to the sampling data. The double scaling is evident also for the sampling points. Note that $CV^2$ has got a lower bound at $0$ which corresponds to the less variable case: all expressions are identical in all samples ($\var{\mathrm{counts}}$). There is an upper bound at $N-1$, with $N$ the number of realizations, that corresponds to the most variable case: a component express in only one realization and is $0$ elsewhere.
\begin{figure}[htb!]
    \centering
    \includegraphics[width=0.9\linewidth]{pictures/scalinglaws/gtex/cvmean_loglog_sampling.png}
    \caption{Coefficient of variation squared versus average. In \textcolor{pythonred}{red} the Poisson-like scaling, in \textcolor{pythongreen}{green} the Taylor-like scaling. In \textcolor{pythonorange}{orange} the sampling components.}
    \label{fig:scalinglaws/gtex/cvmean_loglog_sampling}
\end{figure}

Finally, the data have a double scaling when looking at their global variance across realizations, a Poisson-like where the sampling experimental process is more important and a Taylor-like where the complexity of the data emerges.
Non-coding genes have got low expression and are rare; protein-coding genes, otherwise, express a lot and everywhere and carry more information; this behaviour results in a double scaling. All genes are more variable than a sampling null model and this is the evidence that something interesting is hidden behind the data.


\paragraph{Average versus occurrence}\mbox{}\\
Another interesting analysis can be the relation between the occurrence and the average. In figure~\ref{fig:scalinglaws/gtex/meanDiff_binned_sampling} it is shown the result, it is clear that there is a relation between occurrence and average, genes that express in more realisations (higher occurrence and right in the figure) have an higher average. Moreover doesn't exist genes that have high expression in few realisations; genes that are rare are also difficult to find so have a small average. Note that the average has got a bound due to the fact that counts are integer numbers, so if, for example, one gene express in $n$ of the $R$ samples, it has occurrence $O_i=\frac{n}{R}$ and its average is at least $\avg{\mathrm{counts}}=\frac{1*n}{R}$
\begin{figure}[htb!]
    \centering
    \includegraphics[width=0.9\linewidth]{pictures/scalinglaws/gtex/meanDiff_binned_sampling.png}
    \caption{Relation between the occurrence of a gene and its average across realisations}
    \label{fig:scalinglaws/gtex/meanDiff_binned_sampling}
\end{figure}


\chapter{Topic modelling}\label{ch:topicmodelling}
Once extensively analysed the structure of the dataset, the goal becomes to develop a machine learning method to learn the hidden structure of the data.  
%%intro
Remembering that in chapter~\ref{ch:structure} it emerged some kind of structure behind data, where each tissue seemed to be sampled by a different power law, a topic modelling approach is here proposed.

The idea is that behind data there are hidden variables that describe the relationship between the genes and the samples. Let's call these variables topics.
Firstly it is necessary to build a bipartite network of genes and samples, then nodes are linked considering the gene expression value in the dataset.
\begin{figure}[htb!]
    \centering
    \includegraphics[width=0.7\linewidth]{pictures/topic/bipartite.pdf}
    \caption{An example of a bipartite network. Samples are on the left, genes are on the right. Each link is weighted by gene expression value. On the left side, all nodes of the same colour are clusters of samples. On the right side, all nodes with the same colour are set of genes, also known as topics.\\ 
    Blue lines represent the cluster structure, each blue square is a set of nodes, lines delineate the hierarchical structure.\\
    It is clear in the middle the network separation in genes and samples.}
    \label{fig:topic/bipartite}
\end{figure}

The output of this kind of models consists of sets of genes, the topics, with a probability distribution $P(\text{gene} | \text{topic})$ and probability distributions of these topics inside each sample $P(\text{topic} | \text{sample})$, together they give the relationship between a \textit{sample} and a \textit{gene}.

In this work it is used an innovative and recent approach to topic model, the algorithm was presented by~\cite{gerlach2018network} and well explained in details in~\cite{Peixoto2017}. This model is a sort of stochastic block models~\cite{Holland1983}. 
Topic modelling has being developed and studied to approach linguistics problems, so this algorithm was developed considering words and books in input, links represents the abundance of a word in a book. In chapter~\ref{ch:structure} was evident that there are many similarities between data considered in this work and linguistics' corpora. Referring to data used in this project \textbf{samples} will be the documents and \textbf{genes} will be the words.
It is expected that topics represent some properties of the system due to the gene expression distribution in samples.

The ultimate goal would be to be able to separate healthy and diseased samples then separate and find well-known tumour types, then extend the actual knowledge and retrieve the tumour sub-types.

One of the advantages of this particular algorithm is that it is hierarchical, so it applies community detection at different layers. So the output has got different resolutions and number of clusters found at each layer. One extreme layer is the one which separates genes ($\simeq$ words) and samples ($\simeq$ samples), by definition; in other layers it is possible to have few big clusters until the other extreme were the number of clusters is comparable with the number of nodes.
\begin{figure}[htb!]
  \centering
  \includegraphics[width=0.6\linewidth]{pictures/topic/peixioto_hierarchic.jpg}
  \caption{Example of a hierarchical structure. At $l=0$ the number of cluster is comparable with the number of nodes, is the situation with many small clusters.}
  \label{fig:topic_peixioto_hierarchic}
\end{figure}

What the algorithm does is to run a sort of Monte Carlo simulation and find the best partition of the data.
The probability that the hidden variables $\theta$ describe the data $G$ $P(\theta | G)$ can be written as a likelihood times a prior probability as 
\[P(\theta|G)=\frac{P(G|\theta)\overbrace{P(\theta)}^{prior}}{\underbrace{P(G)}_{\sum_{\theta}P(G|\theta)P(\theta)}}.\]
It is possible to define a description length
\[
\Sigma=-lnP(G|\theta)-lnP(\theta),
\]
so that $P(\theta | G)\propto e^{-\Sigma}$.
Moreover, the likelihood $P(G | \theta)$, can be written as $\frac{1}{\Omega}$ where $\Omega$ is the number of possibles realizations given $\theta$. This can be represented as a microcanonical ensemble with entropy $S=Ln\left(\Omega\right)$. According to~\cite{peixoto2017nonparametric} entropy $S$ can be written as
\[
S=\frac{1}{2}\Sigma_{r,s} n_r n_s H\left(\frac{e_{rs}}{n_rn_s}\right),
\]
where $n_r$ is the number of nodes in block $r$, $e_{rs}$ the number of links between nodes of group $r$ and group $s$ and $H$ is the Shannon entropy $H(x)=xLog_2(x)+(1-x)Log_2(1-x)$. Note that $S$ is minimal if $\frac{e_{rs}}{n_rn_s}$ is close to zero, $r$ and $s$ are two completely separated blocks or if it is close to $1$, $r$ and $s$ are groups with many connections; this allows to find groups with nodes very disconnected or topic and clusters with a lot of connections. Note that the description length depends on the entropy:
\[
\Sigma=S-lnP(\theta),
\]
The algorithm tries to minimize $S$, so that $\Sigma$ is minimized, so $e^{-\Sigma}$ is maximized, but this is $P(\theta | G)$ that is the required probability to maximize.

The Monte Carlo works in a few steps:
\begin{itemize}
 \item a node $i$ is chosen
 \item the group of $i$ is called $r$
  \item a node $j$ is chosen from $i$'s neighbours, the group of $j$ is called $t$
  \item a random group $s$ is selected
  \item move of node $i$ to group $s$ is accepted with probability $P(r\to s|t)=\frac{e_{ts}+\epsilon}{e_t+\epsilon B}$
  \item if $s$ is not accepted, a random edge $e$ is chosen from group $t$ and node $i$ is assigned to the endpoint of $e$ which is not in $t$
\end{itemize}
in figure~\ref{fig:topic_peixioto_move} an example of these steps.
\begin{figure}[htb!]
  \centering
  \includegraphics[width=0.9\linewidth]{pictures/topic/peixioto_move.jpg}
  \caption{Left: Local neighbourhood of node $i$ belonging to block $r$, and a randomly chosen neighbour $j$ \
  belonging to block $t$. \
  Right: Block multi graph, indicating the number of edges between blocks, represented as the edge thickness. \
  In this example, the attempted move $bi \to s$ is made with a larger probability than either $bi \to u$\
   or $bi \to r$ (no movement), since $e_{ts}>e_{tu}$ and $e_{ts}>e_{tr}$.}
  \label{fig:topic_peixioto_move}
\end{figure}
Once the model run it is possible to estimate the probability distribution of words inside a topic
\[P(w|t_w)=\frac{\text{\# of edges on $w$ to $t_w$}}{\text{\# of edges on $t_w$}}\]
and the topic distribution inside a document
\[P(t_w|d)=\frac{\text{\# of edges on $d$ from $t_w$}}{\text{\# of edges on $d$}}\]
In the case of overlapping partitions, the presence of a word in a topic is not trivial and can be estimated as 
\[P(t_w|w)=\frac{\text{\# of edges on $w$ to $t_w$}}{\text{\# of edges on $w$}}\]
or the presence of a document in a cluster
\[P(t_d|d)=\frac{\text{\# of edges on $d$ to $t_d$}}{\text{\# of edges on $d$}}\]

See appendix~\ref{app:hsbm} for detailed analysis of the maths behind the algorithm and \url{https://cloud.docker.com/repository/docker/fvalle01/hsbm} for the extension of~\cite{gerlach2018network} to non linguistics component systems datasets.


%%metrics
\section{Test the model with metrics and benchmarks}
Before running topic modelling, it is useful to define some metrics to test and benchmark the model. In particular the model searches sets on the two sides of the network: the one containing samples and the one containing genes. Samples are extracted from datasets where much metadata are available, some of these metadata labels will be used to benchmark the model. To study genes, enrichment test are instead necessary.

Looking at the samples side of the network, the outputs are sets of samples, let's call these clusters. One can state that the model works if all, or at least the majority, of samples in the same cluster share some label. Here the sample primary site is considered as the main label.

Note that this work's model is a non-supervised one, but a ground truth is available from metadata. So every sample has a certain probability to have a certain property (the true tissue label), let's call this $P(C)$ and a certain probability of being in a cluster (model's output), let's call this $P(K)$.
It is possible to define, for instance, the homogeneity
\begin{equation}\label{eq:homogeneity}
    h=1-\frac{H(C|K)}{H(C)}
\end{equation}
defining the entropy
\begin{equation}\label{eq:hck}
    H(C|K)=\sum_{c\in \mathrm{labels},\\ k \in \mathrm{clusters}}\frac{n_{c k}}{N}Log\left(\frac{n_{c k}}{n_k}\right)
\end{equation}
where $n_{c k}$ is the number of nodes of type $c$ in cluster $k$, $N$ the number of nodes and $n_k$ the number of nodes in cluster $k$. It is evident that if all nodes inside cluster $k$ are of the same type $c$ $n_{c k}=n_{k}$, $H(C|K)=0$ and $h=1$, it is actually a full homogeneous situation.

Another quantity can be defined: the so-called completeness
\begin{equation}\label{eq:completness}
    c=1-\frac{H(K|C)}{H(K)},
\end{equation}
$H(K|C)$ is defined in the same way as~\ref{eq:hck}. Completeness measures how well nodes of the same type are distributed in the same cluster.

Ideally one wants a method which output is both homogeneous and complete. So it is possible to define the V-measure as the harmonic average of the two:
\begin{equation}\label{eq:mutualinformation}
    \mathrm{V-measure}=2\frac{h c}{h + c},
\end{equation}
which is actually the normalized mutual information between $P(C)$ and $P(K)$~\cite{rosenberg2007v}. Please refer to appendix~\ref{app:vmeasure} for the detailed maths. In figure~\ref{fig:topic/cartoon/hc_table} a simplified example of the homogeneity and completenesses ideas. ~\cite{Shi} proposed a similar metric to compare topic modelling algorithms' performances.
\begin{table}[htb!]
	\centering
	\begin{tabular}{|c|c|c|}
		\hline
		&Homogeneous & Not homogeneous\\ \hline
		\rotatebox[origin=l]{90}{Complete}&    \includegraphics[width=0.2\textwidth]{pictures/topic/cartoon/cartoon_hc.pdf}&\includegraphics[width=0.2\textwidth]{pictures/topic/cartoon/cartoon_c.pdf}  \\ \hline
		\rotatebox{90}{Not complete}&   \includegraphics[width=0.2\textwidth]{pictures/topic/cartoon/cartoon_h.pdf}&\includegraphics[width=0.2\textwidth]{pictures/topic/cartoon/cartoon.pdf} \\ \hline
	\end{tabular}
	\caption{Examples of homogeneity and completeness. Homogeneous clusters contain all nodes with the same label. A label is complete if it is fully represented by a single cluster. In this image some extreme examples of these definitions.}
	\label{tab:topic/cartoon/hc_table}
\end{table}
			
In figure~\ref{fig:topic/metric_scores_primarysite} an example of the V-measure score estimated at the different layers of the hierarchy; note that the number of clusters increases going deeper in the hierarchy. In the same figure homogeneity and completeness are reported, note that with few clusters the situation is more complete, but when the number of clusters increases completeness goes down and homogeneity increases. This happens because if a cluster is small it is easier to fulfil it with similar objects. On the other side if one has few clusters it is easier to complete them putting similar objects in the same cluster.
\begin{figure}[htb!]
    \centering
    \includegraphics[width=0.8\linewidth]{pictures/topic/gtex/oversigma_10tissue/metric_scores_primarysite.pdf}
    \caption{Score across hierarchy. The V-measure or normalized mutual information MI is the harmonic average between homogeneity and completeness.}
    \label{fig:topic/metric_scores_primarysite}
\end{figure}

In order to validate the model, it will be compared to more standard approaches such as standard hierarchical clustering and a classical topic model approach using Latent Dirichlet Allocation. 
\FloatBarrier

%%preprocess
\section{Pre-process}
To make the algorithm faster, it could be useful to do a pre-processing of the data.
Different approaches were tested, all of them involving the quantities defined in~\ref{ch:structure}. The goal is to identify components which are able to best separate the realizations. 
\paragraph{Low occurrence genes} were selected firstly to approach topic modelling. A $0.5$ threshold was set on occurrence. This method selects genes that appears (have expression greater than zero) only in less than half samples. This approach has some limitations, for instance it doesn't consider genes that appear everywhere (with occurrence $\simeq 1$) but changes their behaviour across realisations.

\paragraph{tf-idf (term frequency–inverse document frequency)} should help. This approach doesn't take in account original expression values $n_{ij}$, but a transformed version
\[
n^{new}_{ij}=\frac{n_{i j}}{M_j}\times \left(1-Log\left(o_i\right)\right)
\] which increases the importance of components with small occurrence $o_i$. This approach doesn't actually select components, which is still an issue.

\paragraph{Highly variable} genes can be selected. This is done using the $CV^2$ analysis done in chapter~\ref{ch:scalinglaws}.
\begin{figure}[htb!]
    \centering
    \includegraphics[width=0.8\linewidth]{pictures/topic/cvmean_oversigma.png}
    \caption{Highly variable genes}
    \label{fig:topic/cvmean_oversigma}
\end{figure}
Plotting the coefficient of variation versus the mean for each component reveals which components have higher variance with respect to components which, on average, have a similar behaviour. Binned averages and variances were estimated, and only genes with a $CV^2$ over a $\sigma$ greater than the bin's mean were considered. This method seems to select useful genes even if the binned average bound is quite noisy.

\paragraph{Distance from boundaries} can be a similar and alternative method to select highly variable genes. In this case the bound is smooth and well defined.
\begin{figure}[htb!]
    \centering
    \includegraphics[width=0.8\linewidth]{pictures/topic/cvmean_oversampling.png}
    \caption{Genes distant from the boundaries}
    \label{fig:topic/cvmean_oversampling}
\end{figure}
The distribution as discussed in~\ref{ch:scalinglaws} have a Poisson-like and a Taylor-like boundaries. So can be considered only components that are the most distant from these boundaries. Moreover this boundaries can be found with a simple null model, as shown in figure~\ref{fig:scalinglaws/gtex/cvmean_loglog_sampling} the sampling model defines the lower bound of the data.

The last two approaches are the ones which lead to better results, in the following sections gene selection was done by getting only highly variables genes.

\section{Run on GTEx}
\draft{Firstly the algorithm is run on a subset of $5$ tissues of GTEx}
\draft{metti 5 tissues?}
\begin{figure}[htb!]
    \centering
    \includegraphics[width=0.9\linewidth]{pictures/topic/gtex/oversigma_10tissue/metric_scores.pdf}
    \caption{Scores accross hierarchy}
    \label{fig:my_label}
\end{figure}

\begin{figure}[htb!]
    \centering
    \includegraphics[width=0.9\linewidth]{pictures/topic/gtex/oversigma_10tissue/clustercomposition_l3_primary_site.pdf}
    \caption{Caption}
    \label{fig:topic/gtex/oversigma_10tissue/clustercomposition_l2_primary_site}
\end{figure}

\begin{figure}[htb!]
    \centering
    \includegraphics[width=0.9\linewidth]{pictures/topic/gtex/oversigma_10tissue/fraction_clustercomposition_l3_primary_site.pdf}
    \caption{Caption}
    \label{fig:topic/gtex/oversigma_10tissue/fraction_clustercomposition_l2_primary_site}
\end{figure}

\begin{figure}[htb!]
    \centering
    \begin{minipage}{0.45\textwidth}
    \includegraphics[width=0.9\linewidth]{pictures/topic/gtex/oversigma_10tissue/shuffledcluster_maximum_l0_primary_site.pdf}
    \end{minipage}
    \hspace{3mm}
    \begin{minipage}{0.45\textwidth}
    \includegraphics[width=0.9\linewidth]{pictures/topic/gtex/oversigma_10tissue/shuffledcluster_maximum_l1_primary_site.pdf}
    \end{minipage}
    \\
    \begin{minipage}{0.45\textwidth}
    \includegraphics[width=0.9\linewidth]{pictures/topic/gtex/oversigma_10tissue/shuffledcluster_maximum_l2_primary_site.pdf}
    \end{minipage}
    \hspace{3mm}
    \begin{minipage}{0.45\textwidth}
    \includegraphics[width=0.9\linewidth]{pictures/topic/gtex/oversigma_10tissue/shuffledcluster_maximum_l3_primary_site.pdf}
    \end{minipage}
\end{figure}


\begin{figure}[htb!]
    \centering
    \includegraphics[width=0.9\linewidth]{pictures/topic/gtex/oversigma_10tissue/fraction_clustercomposition_l2_primary_site.png}
    \caption{Caption}
    \label{fig:topic/gtex/oversigma_10tissue/fraction_clustercomposition_l2_primary_site}
\end{figure}

\begin{figure}[htb!]
    \centering
    \includegraphics[width=0.9\linewidth]{pictures/topic/gtex/oversigma_10tissue/fraction_clustercomposition_l2_secondary_site.png}
    \caption{Caption}
    \label{fig:topic/gtex/oversigma_10tissue/fraction_clustercomposition_l2_secondary_site}
\end{figure}


Using as gene set ~\cite{Ardlie2015} enrichment test can be made \cite{Kuleshov2016}

Enrichment test are made once for each topic, starting from the layer with more genes per 
single topic. Test are made across multiple categories.

\section{Run on TCGA}

\section{Mixed runs}

\draft{hierarchical clustering}
\draft{This is better than LDA because...}

\chapter{Results}\label{ch:results}

Using as gene set ~\cite{Ardlie2015} enrichment test can be made \cite{Kuleshov2016}

Enrichment test are made once for each topic, starting from the layer with more genes per 
single topic. Test are made across multiple categories.


\chapter{Conclusions}\label{ch:conclusions}
Finally this work demonstrates that RNA-Sequencing datasets can be analyses from a component systems point of view.
This kind of data shows typical trends famous, for example, in linguistics, moreover some interesting biological signatures were found. RNA-Seq dataset have a great core of protein coding genes that express everywhere, this is evident looking at $U$s, Heaps' law. The presence of a power law distribution in the ranked abundances, the so called Zipf's law is observed and characterize the distribution of genes expression data.

In the first part of this work a dataset (GTEx) containing samples from healthy tissues was analysed. One of the most interesting evidences was the presence of many different Zipf's law if one considers each tissue independently. Very similar results were obtained considering TCGA, a dataset containing thousands of samples of cancer tissues.

The power law distribution encouraged to explore the possibility of using a topic model approach to reveal the hidden structure of these datasets. This approach is useful both to find clusters of samples that share some properties and to find the relation between genes and samples.

Many goals were achieved during these analysis.
First of all it was developed a pipeline that begins with creating a network with useful genes and samples, this network is than processed with hierarchic stochastic block model topic model algorithm and then analysed.


In conclusion topic model reveal itself as a useful approach to this kind of data.

Verified that benchmark give good results the next goal would be to 

\draft{future: Loredana, Jacopo, topic on sampling, Jonathan}

%%APPENDIXES
\appendix
\addcontentsline{toc}{chapter}{Appendices}
\chapter{Hierarchical stochastic block model}\label{app:hsbm}
The algorithm is called hierarchic Stochastic Block Model.

The first step of hierarchical stochastic block model, as discussed in~\cite{peixoto2014efficient},
is to create a bipartite network $G$ with two kind of nodes: \textbf{words} and \textbf{documents}.
Every time a word $w$ is present in a document $d$ an edge $e_{wd}$ is created.
If a word count in the entire corpus is under a certain threshold, that word is ignored.
The aim is to find a partition $b\in\{b_i\}$ with $B=\left|\{b_i\}\right|$ blocks.

These kind of models are called \textit{generative models}: given the data the model
should generate a network $G$ with probability $P(G|\theta, b)$, where $b$ is
the partition and $\theta$ any additional parameter of the model.

Using well-known Bayes theorem one could estimate the probability that an
observed network is generated by partition $b$
\begin{equation}\label{eq:PbonG}
  P(b,\theta|G)=\frac{P(G|b,\theta)\overbrace{P(b,\theta)}^{prior}}{\underbrace{P(G)}_{\sum_{\theta}P(G|\theta, b)P(\theta, b)}}
\end{equation}

defining the amount of information needed to describe the data as the description length
\begin{equation}\label{eq:descriptionlenght}
  \Sigma=-lnP(G|b,\theta)-lnP(b, \theta)
\end{equation}
the~\ref{eq:PbonG} can be written as $\frac{e^{-\Sigma}}{P(G)}$, so maximising that is equivalent to minimise the description length~\ref{eq:descriptionlenght}. The probability of obtaining a Graph from a set of parameters is $P(G|b,\theta)=\frac{1}{\Omega(A,\{n_r\})}$, where $\Omega(A,\{n_r\})$ is the number of graph that is possible to generate with audience matrix $A$ and $n_r$ the counts of block partition $\{b_i\}$

In case of a weighted network the likelihood becomes $P(G,x|b,\theta)$, where $x$
are the weights.

\paragraph{Algorithm}
First of all a $B\times B$ matrix is created. The entry $e_{rs}$ of this matrix represents the number of links between nodes of group $r$ and nodes of group $s$, with $r,s\in\{b_i\}$. At the beginning $B$ groups are formed at random and the initial $B$ is a hyper-parameter of the model.
\begin{figure}[htb!]
  \centering
  \includegraphics[width=0.3\linewidth]{pictures/topic/peixioto_ers.pdf}
  \caption{Example of a edge's matrix from~\cite{peixoto_graph-tool_2014}}
    \label{fig:hsbm-ers}
\end{figure}

It is useful to define a traditional entropy:
\begin{equation}\label{eq:hSBMentropyt}
  S_t=\frac{1}{2}\Sigma_{r,s} n_rn_sH\left(\frac{e_{rs}}{n_rn_s}\right)
\end{equation}
where $n_{r}$ is the number of nodes in groups $r$, $e_{rs}$ is the
number of edges between nodes of group $r$ and nodes of group $s$, and
$H(x)=-xln(x)-(1-x)ln(1-x)$. This entropy is equivalent to the micro-canonical
entropy of a system with ${\Omega(A,\{n_r\})}$ accessible states $S_t=Ln\Omega$.

The algorithm uses a Markov Chain Monte Carlo to minimise this entropy.
At each step a node changes block and the new configuration is accepted if $S$ is decreased.

Note that~\ref{eq:hSBMentropyt} can be corrected taking care of degree
distribution obtaining corrected entropy $S_c$
\begin{equation}
  S_c=-\Sigma_{r,s}\frac{e_{rs}}{2}-\Sigma_k
  N_kln(k!)-\frac{1}{2}\Sigma_{r,s}e_{rs}ln\left(\frac{e_{rs}}{e_re_s}\right)
\end{equation}

\paragraph{How to change group of a node?}
At each step according to~\cite{peixoto2014efficient} node $i$ can change group from $r$ to $s$ with a probability
\begin{equation}\label{eq:Prst}
  P(r\to s|t)=\frac{e_{ts}+\epsilon}{e_t+\epsilon B}
\end{equation}
where $j$ is a random neighbour of $i$: $j\in N_i$, $t\in\{b_j\}$ its block as defined in~\cite{peixoto2014efficient}.
$\epsilon$ is a parameter that according to~\cite{peixoto2017nonparametric} has no significant impact in the algorithm,
provided it is sufficiently small.

~\ref{eq:Prst} can be rewritten as \[P(r\to s|t)=(1-R_t)\frac{e_{ts}}{e_t}+\frac{R_t}{B}\]
defining $R_t=\frac{\epsilon B}{e_t + \epsilon B}$

This is done in four steps for each node $i$:
\begin{itemize}
  \item a node $j$ is chosen from $i$'s neighbours, the group of $j$ is called
  $t$
  \item a random group $s$ is selected
  \item move of node $i$ to group $s$ is accepted with probability $R_t$
  \item if $s$ is not accepted, a random edge $e$ is chosen from group $t$ and node $i$ is assigned to the endpoint of $e$ which is not in $t$
\end{itemize}
This steps mime probability~\ref{eq:Prst}; note that for $\epsilon\to\infty$ this gives a uniform probability.

To enchant the probability to go into a minimum, a bounce of these moves is made, only the set of moves with the minimum $S$ is accepted.
\paragraph{How many blocks $B$?}
Note that the number of blocks $B$ is a free parameter and must be inferred as described in~\cite{peixoto2017nonparametric}.
This implies a slight modification of the algorithm such that
it became possible to admit that a new group is created.
When a group $s$ is chosen, the algorithm can now accept a \textbf{new group} and~\ref{eq:Prst} became
\begin{equation}\label{eq:PrstB1}
  P(r\to s)=\Sigma_t P(t|i)\frac{e_{ts}+\epsilon}{e_t+\epsilon (B+1)}
\end{equation}
being $P(t|i)=\Sigma_j\frac{A_{ij}\delta(b_j, t)}{k_i}$ the fraction of neighbours of $i$ belonging to group $t$, $e_t$ the number of edges in group $t$,
$k_i$ the degree, and $b_j$ groups.

Using this modification it is now possible to add new groups and $B$ is no longer a parameter.

\paragraph{How to find hierarchic layers?}
After the algorithm is run, one may would to add a new hierarchic level, this is done considering the $B$ groups as nodes and repeating the process.
As done before a matrix of edges like~\ref{fig:hsbm-ers} is created, where edges
are considered between groups of the previous layer.

The posterior probability became
\begin{equation}\label{eq:posteriorL}
  P(\{b_l\}|A)=\frac{P(A|\{b_l\})P(\{b_l\})}{P(A)}=\prod_l^L P(b_l|e_l,b_{l-1})
\end{equation}
where $l=0\dots L$ is the layer, $A$ the audience matrix, $b_i$ blocks. Note that $e_0=A$ and $B_L=1$.
Maximising~\ref{eq:posteriorL} gives the correct number of layers.

Adding a layer is done in 3 steps described in~\cite{peixoto2014hierarchic}:
\begin{itemize}
  \item[Resize] find $B_l\in[B_{l-1},B_{l+1}]$ by bisection
  \item[Insert] a layer l
  \item[Delete] $l$ and linking nodes from layer $l-1$ directly to groups of layer
  $l+1$
\end{itemize}
One marks initially all levels as not done and starts at the top level $l = L$~\cite{peixoto2014hierarchic}.
For the current level $l$, if it is marked done it is skipped and one moves to the level $l-1$.
Otherwise, all three moves are attempted. If any of the moves succeeds in decreasing the description length $\Sigma$~\ref{eq:descriptionlenght},
one marks the levels $l-1$ and $l+1$ (if they exist) as not done, the level $l$ as done, and one proceeds (if possible)
to the upper level $l+1$, and repeats the procedure.
If no improvement is possible, the level $l$ is marked as done and one proceeds to the lower level $l-1$.
If the lowest level $l=0$ is reached and cannot be improved, the algorithm ends.

%%done by heuristic agglomerative

\paragraph{Overlapping partitions}
As described in~\cite{peixoto2015model} one of the advantages of this approach
is that it is possible to let a node belonging to multiple groups.
In this case $b_i$ becomes $\vec{b_i}$, with component $b_{ir}=1$ if node $i$ is in
group $r$, $0$ otherwise. The number of $1$s in vector $\vec{b_i}$ is called
$d_i=|\vec{b_i}|$.

The probability of having a graph $G$ being generated from an audience matrix
$A$ and a partition $\{\vec{b_i}\}$ is \[P(G|A,\{\vec{b_i}\})=\frac{1}{\Omega}\]
if $\Omega$ is the number of possible graphs. Entropy~\ref{eq:hSBMentropyt} is
$S_t=Ln\Omega$. This corresponds to an augmented graph generated via a
non overlapping block model with $N'=\Sigma_r n_r>N$ nodes and the same
audience matrix $A$.

First of all, it is necessary to sample the distribution of mixture sizes
$P(\{n_d\})$ where $n_d$ is the number of nodes which mixture has got size $d$, $n_d\in[0,N]$ and $d\in[0,D]$ (typically $D=B$
and in the non-overlapping case $D=1$), this is done by sampling uniformly from
\[P(\{n_d\}|B)=\left(\binom{D}{N}\right)^{-1}\] which is probability of having $n$ nodes whose mixture has size $d$.
$\left(\binom{B}{N}\right)$ is the number of histograms with
area $N$ and $B$ distinguishable bins. $B-1$ can be used instead of $B$ to avoid node with no group, in this case $d\in[1,B]$.

Given the mixture sizes, the distribution of node membership
is sampled from \[P(\{d_i\}|\{n_d\})=\frac{\prod_{d} n_d!}{N!}\].

At this point for each set of nodes with $d_i=d$ it is necessary to sample $n_{\vec{b}}$; the number of nodes
with a particular mixture $\vec{b}$.
It is sampled from
\begin{equation}
  P(\{n_{\vec{b}}\}_d|n_d)=\left(\binom{\binom{D}{d}}{n_d}\right)^{-1},
\end{equation}
next all mixtures $\vec{b_i}$ of size $d$ must be sampled, they are given by
\begin{equation}
  P(\{\vec{b_i}\}_d|\{n_{\vec{b}}\}_d)=\frac{\prod_{|\vec{b_i}|=d} n_b!}{n_d!}
\end{equation}
the global posterior as defined in~\cite{peixoto2015model} is
\begin{equation}
  P(\{\vec{b_i}\}|B)=\left[\prod_{d=1}^B  P(\{\vec{b_i}\}_d|\{n_{\vec{b}}\}_d) P(\{n_{\vec{b}}\}_d|n_d)\right]P({d_i}|{n_d})P(n_d|B)
\end{equation}

At this time it is necessary to obtain the distribution of the edges between
mixtures. Defined $e_r=\Sigma_s e_{rs}$ the number of half-edges labelled $r$,
$m_r=\Sigma_{\vec{b}} b_r$ the number of mixtures containing group $r$ the
algorithm samples the probability distribution of the edges count
\[P(\{e_{\vec{b}}\}|\{\vec{b_i}\}, A)=\prod_r\left(\binom{m_r}{e_r}\right)^{-1}\] and the
labelled degree sequence $\{\vec{k_i}\}$ from
\[P(\{\vec{k_i}\}_{\vec{b}}|\{e_{\vec{b}}\}, \{\vec{b_i}\})=\frac{\prod_k n_k^{\vec{b}}!}{n_{\vec{b}}!}\]

\paragraph{Word documents separation}
Following what is done in~\cite{gerlach2018network}, the probability of a group $P(b_l)$ at a certain level $l$ is intended as the disjoint probability of
group of words and group of documents.
\begin{equation}
  P(b_l)=P_w(b_l^w)P_d(b_l^d)
\end{equation}
Doing this let words and documents be separated by construction.
Considering the process described above if two nodes are not connected at the beginning it is impossible
that they end up in the same block.
It is easily verified in~\cite{peixoto2014efficient} that this property is preserved and fully reflected in the final block structure.

\chapter{Homogeneity, completeness and V-measure}\label{app:vmeasure}
Using algorithms that are unsupervised, but with a ground truth available it is useful to define some metrics.

The homogeneity
\begin{equation}
    h=1-\frac{H(C|K)}{H(C)}
\end{equation}
defining the entropy
\begin{equation}
    H(C|K)=\sum_{c\in \mathrm{model labels},\\ k \in \mathrm{clusters}}\frac{n_{c k}}{N}Log\left(\frac{n_{c k}}{n_k}\right)
\end{equation}
where $n_{c k}$ is the number of nodes of type $c$ in cluster $k$, $N$ the number of nodes and $n_k$ the number of nodes in cluster $k$. It is evident that if all nodes inside cluster $k$ are of the same type $c$ $n_{c k}=n_{k}$, $H(C|K)=0$ and $h=1$, it is actually a complete homogeneous situation.
The completeness:
\begin{equation}
    c=1-\frac{H(K|C)}{H(K)},
\end{equation}
$H(K|C)$ is defined in the same way as $H(C|K)$. Completeness measures if all nodes of the same type are in the same cluster.
Ideally one wants a model which output is both homogeneous and complete. So it is possible to define the V-measure~\cite{rosenberg2007v}, which is the harmonic average of the two:
\begin{equation}
    2\frac{h c}{h + c}.
\end{equation}

The product $h c$ is equal to
\begin{equation}
    \frac{(H(C)-H(C|K))(H(K)-H(K|C))}{H(K) H(C)},
\end{equation}
the sum $h + c$ is
\begin{equation}
    \frac{H(K)(H(C)-H(C|K))+H(C)(H(K)-H(K|C))}{H(K) H(C)}.
\end{equation}
Expressing the conditional entropy 
\[
H(K|C)=\sum_{k c} P(k,c)Log(P(k|c))
\\=\sum_{k c} P(k,c)Log\left(\frac{P(k,c)}{P(c)}\right)
\\=H(K,C) - H(C)
\]
in terms of the conjunct entropy $H(K,C)$ which is symmetric by exchanges of $C$ and $K$
\[
H(K,C)=H(K|C) + H(C) = H(C|K) + H(K) = H(C,K)
\]
it is easy to verify that 
\[
H(C) - H(C|K) = H(K) - H(K|C) 
\]
so
\[
h c = \frac{(H(C)-H(C|K))^2}{H(K) H(C)}
\]
and
\[
h + c = \frac{(H(C)-H(C|K))(H(K)+H(C))}{H(K) H(C)}.
\]
The harmonic average $2\frac{h c}{h + c}$ becomes
\[
2\frac{H(C)-H(C|K)}{H(K)+H(C)}=2\frac{H(C)+H(K)-H(K,C)}{H(K)+H(C)}=2\frac{MI(C,K)}{H(K)+H(C)}
\]
which is called V-measure and is actually the mutual information between $P(C)$ and $P(K)$ normalised to $1$ by the term $H(C)+H(K)$. In fact if $P(C)=P(K)$ $H(K,C)=H(K)=H(C)$ and the measure is $1$, if $P(C)$ and $P(K)$ are completely independent $H(K,C)=H(K)+H(C)$ and the measure is $0$.


%%%% TAIL OF THE DOCUMENT
\backmatter
%list of figures
\listoffigures
\clearemptydoublepage
%list of tables
\listoftables
\clearemptydoublepage

%bibliography
\addcontentsline{toc}{chapter}{Bibliography}
\bibliography{bibliography/bibThesis}
\bibliographystyle{ieeetr}
\clearemptydoublepage

\addcontentsline{toc}{chapter}{Acknowledgements}
\chapter*{Acknowledgements}
\draft{compila l'altro..}

Infine ringrazio l'intero gruppo ByoPhys~\url{http://personalpages.to.infn.it/~caselle/BioPhys/BioPhys.html}: Michele Prof. Caselle e Matteo Dott. Osella per avermi instradato a fare questo lavoro e per avermi supportato anche molto oltre le loro aree di competenza.
Francesco, Marta, Mattia, Eleonora, Marco, Serena, Gabriele è stato bello lavorare con voi, ci vediamo al PhD!


\end{document}
