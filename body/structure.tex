\chapter{Data structure}\label{ch:structure}

The data studied in this work can be represented as component systems. These component systems can be represented by a two dimensional matrix in which rows represent components and columns are the possible realizations buildable given subset of the components. The entries of this matrix are the number of the components on the row needed during the realization of the column. In figure~\ref{fig:componetstable} an example of this kind of matrices.

%%data definitions
\section{Component systems}
The most common example of such systems is a set of books. In this case one puts on the rows the words in the whole vocabulary and the books' titles on the columns. The entry that corresponds to row $i$ and column $j$ is the number of times the word $i$ appears in the book $j$. The same happens if one considers Wikipedia's pages. Other examples are: Lego$\textsuperscript{\tiny\textregistered}$ sets where components are the Lego$\textsuperscript{\tiny\textregistered}$ bricks and realizations the Lego$\textsuperscript{\tiny\textregistered}$ packages, and protein domains; all these were described and well studied in~\cite{mazzolini2018heaps, Mazzolini2018zipf}.

Given a matrix with $N$ components on the rows and $R$ realizations on the columns and relative abundances $n_{ij}$ as the entries, it is interesting to study some quantities that are universal and general characteristics of component systems.

First of all, the \textbf{occurrence} of a component is defined as 
\begin{equation}\label{eq:occurrence}
O_i=\frac{\sum_{j=1}^{R}(1-\delta_{n_{ij},0})}{R}.
\end{equation}
It is the fraction of realizations in which the component's abundance is not null. A component that is present in all the realizations has got $O_i=1$, the ensemble of the components with $O_i=1$ is known as the \textbf{core}. Components with high ($\simeq 1$) occurrence are present in mostly all realizations of the datasets. In linguistics articles, such as \textit{the}, are present everywhere, so they have high occurrence. Components with low occurrence ($\simeq 0$) are present only in a few realizations and are the most specific ones~\cite{altmann2016statistical}.

The sum across all columns, or the number of times component $i$ appear in the dataset, is called \textbf{abundance} of the component and is defined as
\begin{equation}\label{eq:abundance}
a_i=\sum_{j=1}^{R}n_{ij};
\end{equation}
dividing this by the global abundance, or the total number of components in the dataset,
\begin{equation}
  a=\sum_{i=1}^{N}a_i
\end{equation}
naturally brings to the \textbf{frequency of a component} in the whole corpus
\begin{equation}\label{eq:fi}
f_i=\frac{a_i}{\sum_{c=1}^{N}a_{c}}.
\end{equation}
The abundance of a component divided by the sum of all the abundances in a realization gives the \textbf{frequency} of the component in that specific realization
\begin{equation}
f_{ij}=\frac{n_{ij}}{\sum_{c=1}^{N}n_{cj}}.
\end{equation}

The sum of all abundances in a realization,
\begin{equation}\label{eq:size}
M_j=\sum_{c=1}^{N}n_{cj}
\end{equation}
represents the \textbf{size} of the realization. In gene expression this is the size of the transcript.

It is expected that frequencies distribute according to the so-called Zipf's law
\begin{equation}\label{eq:zipf}
f_i\propto r_i^{-\alpha}
\end{equation}
where $r$ is the rank: the position of a component when sorting, in descending order, all components by their frequencies in the whole dataset.

%%universal laws
\section{Universal laws in RNA-Seq}
\subsection{TCGA}
Analysing TCGA dataset~\cite{grossman2016toward} the first interesting analysis is to plot the sorted abundance, this gives the so called Zipf's law. The analysis were made considering \textit{Gene Expression Quantification} as data type, \textit{Transcriptome Profiling} as data, \textit{RNA-Seq} as experimental strategy, \textit{HTSeq - Counts} or \textit{HTSeq - FPKM} as workflow type. $5000$ samples were downloaded and analysed.
\begin{figure}[htb!]
    \centering
    \begin{minipage}{0.45\textwidth}
    \includegraphics[width=0.95\linewidth]{pictures/structure/tcga/globalzipf_fpkmall.pdf}
    \end{minipage}
\hspace{3mm}
    \begin{minipage}{0.45\textwidth}
    \includegraphics[width=0.95\linewidth]{pictures/structure/tcga/globalzipf_fpkm.pdf}
    \end{minipage}
    \caption{Zipf's law from FPKM normalised data. On the right considering only protein coding genes}
    \label{fig:structure/tcga/globalZipf}
\end{figure}
In figure~\ref{fig:structure/tcga/globalZipf} it is shown the frequency ranked plot. It is interesting that this kind of data distribute according a power law with exponent close to $1$, this same behaviour can be found in completely different systems such as linguistics' ones~\cite{altmann2016statistical}. Another interesting fact is that considering in the analysis also non-coding genes gives a double-scaled power law. This is due to the fact that non coding genes are also more specific and rare, so their frequencies are quite small compared to protein coding genes.

Changing normalisation and considering counts instead of FPKM, the result is quite similar. The power law is more flat, meaning that genes have more similar abundances in the whole dataset. 
\begin{figure}
    \centering
    \includegraphics[width=0.8\linewidth]{pictures/structure/tcga/globalzipf_counts.pdf}
    \caption{Zipf's law of protein coding genes considering counts}
    \label{fig:structure/tcga/globalzipf_count}
\end{figure}

\subsection{GTEx}
A pretty similar analysis can be made on GTEx's~\cite{carithers2015novel} healthy samples. RNA sequencing raw counts data were download from file version \textit{2016-01-15 v7 RNASeQCv1.1.8}. All $\sim 11000$ samples available were considered at this time.

\begin{figure}[htb!]
    \centering
    \begin{minipage}{0.45\textwidth}
    \includegraphics[width=0.95\linewidth]{pictures/structure/gtex/globalZipf.pdf}
    \end{minipage}
    \hspace{3mm}
    \begin{minipage}{0.45\textwidth}
    \includegraphics[width=0.95\linewidth]{pictures/structure/gtex/globalZipf_c.pdf}
    \end{minipage}
    \caption{Zipf's law from GTEx count data. On the left all genes considered, on the right only protein coding ones}
    \label{fig:my_label}
\end{figure}
Not surprisingly in the GTEx dataset it is retrieved the same behaviour at this time. The power law with exponent $\simeq 1$ is found and considering non coding genes gives to a knee in the power law.

Going further in the analysis it is possible make an histogram of occurrences defined by~\ref{eq:occurrence}, also known as $U$s.
\begin{figure}[htb!]
    \centering
    \includegraphics[width=0.9\linewidth]{pictures/structure/gtex/U_gtex_cnc.pdf}
    \caption{The histogram of the occurrences $O_i$}
    \label{fig:structure/gtex/U_cnc}
\end{figure}

Also in this kind of analysis it is possible to see the different behaviour of coding and not coding genes. The protein coding genes express almost in every sample, so their occurrence is near to $1$, non coding genes are more specific, so they are present only in a subset of the dataset and many of the have small occurrence.
\begin{figure}[htb!]
    \centering
    \begin{minipage}{0.45\textwidth}
    \includegraphics[width=0.95\linewidth]{pictures/structure/gtex/U_Breast.png}
    \end{minipage}
    \hspace{2mm}
    \begin{minipage}{0.45\textwidth}
    \includegraphics[width=0.95\linewidth]{pictures/structure/gtex/U_Brain.png}
    \end{minipage}
    \caption{Same behaviour is observed looking at one tissue a time.}
    \label{fig:structure/gtex/U_tissues}
\end{figure}
The same behaviour can be observed considering just all samples of a given tissue. In this case $O_i=0$ means that the genes has a non zero expression in just one of the samples of the tissue considered; in other words if a gene never express in a tissue it is not considered when constructing these tissue specific $U$ distributions.

From now on except were explicitly declared analysis will be made considering protein coding genes and counts with no normalisation.


%%null model
\section{Null model construction}\label{sec:nullmodel}
The kind of data considered in this work comes from RNA Sequencing experiments. This experiments use wet biology methods to extract information from samples. If one imagines it exists an unknown function that describes the gene expression across the samples considered, what experimenters people do is to sample  this function, picking up some genes.

In this section it is described a null model of sampling, this is useful to verify if the data distributions seen are just an effect of this experimental sample or if they carry some useful and interesting information.

As described in~\cite{mazzolini2018heaps} a random matrix has to be created. This matrix is a collection of components and realizations exactly as~\ref{fig:componetstable}. The values of abundances of each component in each realization $n_{i j}$ are randomly assigned with a probability determined by 
the global abundance in the whole dataset~\ref{eq:abundance}. Values of each column are extracted until the size~\ref{eq:size} is 
reached. Strictly speaking it is a multinomial process
\begin{equation}
P\left({n_i};M\right)=\frac{M!}{\prod_{i=1}^{N} n_i}\prod_{i=1}^N f_i^{n_i}
\end{equation}
where $n_i$ is the number of components with frequency $f_i$, being $f_i=\frac{a_i}{\sum_{i=1}^{N}a_{i}}$ as defined in~\ref{eq:fi}.

Figure~\ref{fig:structure/randomsampling} shows an example of this, $M$ components are picked up with respect to their frequency in the dataset. The most abundant components, which are also the ones with higher frequency (frequency is nothing but the normalised abundance), have a greater probability to be picked up.
\begin{figure}[htb!]
    \centering
    \includegraphics[width=0.8\linewidth]{pictures/structure/randomsampling.png}
    \caption{Random sampling of components to build a realization of size $M$}
    \label{fig:structure/randomsampling}
\end{figure}

Using this construction on data of counts on both dataset, by definition the Zipf's law sampled are identical to the data's one.
\begin{figure}[htb!]
\begin{minipage}{0.5\textwidth}
    \centering
    \includegraphics[width=0.95\linewidth]{pictures/structure/tcga/globalzipf_null.pdf}
\end{minipage}
\hspace{2mm}
\begin{minipage}{0.5\textwidth}
    \centering
    \includegraphics[width=0.95\linewidth]{pictures/structure/gtex/globalzipf_null.pdf}
\end{minipage}
\caption{Zipf's law sampled; TCGA(left) and GTEx (right)}
\label{fig:structure/globalzipf_null}
\end{figure}
By construction the distribution of the sizes of the sampling and of the data are identical.
\begin{figure}[htb!]
\begin{minipage}{0.5\textwidth}
    \centering
    \includegraphics[width=0.95\linewidth]{pictures/structure/tcga/sizeDistr_null.pdf}
\end{minipage}
\hspace{2mm}
\begin{minipage}{0.5\textwidth}
    \centering
    \includegraphics[width=0.95\linewidth]{pictures/structure/gtex/sizeDistr_null.pdf}
    \end{minipage}
\caption{Distribution of sizes $M$; TCGA(left) and GTEx (right)}
    \label{fig:structure/sizeDistr_null}
\end{figure}

Looking at the $U$s, it is evident that data is different from sampling. This is a signal that the null model is not enough to explain the data matrices. In particular from figure~\ref{fig:structure/globalU_null} it is evident that the null model generate the matrices in a manner such that more components have high occurrence with respect to the original data. This can be easily explained, in fact in real world there are some genes that are highly expressed but only in a subset of the whole dataset; these genes are specific for certain type of samples. The null model gets the information the such genes are highly expressed from the abundance and so samples these quite often (components with high abundance have a greater chance to be picked up by the null model sampling).
\begin{figure}[htb!]
\begin{minipage}{0.5\textwidth}
    \centering
    \includegraphics[width=0.95\linewidth]{pictures/structure/tcga/globalU_null.pdf}
\end{minipage}
\hspace{2mm}
\begin{minipage}{0.5\textwidth}
    \centering
    \includegraphics[width=0.95\linewidth]{pictures/structure/gtex/globalU_null.pdf}
    \end{minipage}
\caption{Occurrence distributions; TCGA(left) and GTEx (right)}
\label{fig:structure/globalU_null}
\end{figure}

Looking at the Heaps's law~\cite{Heaps:1978:IRC:539986} 
, again the curves differ and the null model is not enough complete to explain the trend. In figure~\ref{fig:structure/heaps_null} the Heaps's law is presented compared to the one obtained by sampling, note that each data point share the abscissa with a sampling one (figures~\ref{fig:structure/sizeDistr_null} are nothing but the histograms of the abscissas of~\ref{fig:structure/heaps_null}). It happens that the sampling curve is above the data's one. This means that to build a sample of size $M$ just by sampling it is necessary to use a greater number of different genes than the number of different genes actually expressed in nature. In other words in real world are expressed only the genes that are really useful in the sample, and this is not describable just by sampling. This fact is coherent with the fact that the $U$s differ.
\begin{figure}[htb!]
\begin{minipage}{0.5\textwidth}
    \centering
    \includegraphics[width=0.95\linewidth]{pictures/structure/tcga/heaps_null.pdf}
    \end{minipage}
\hspace{2mm}
\begin{minipage}{0.5\textwidth}
    \centering
    \includegraphics[width=0.95\linewidth]{pictures/structure/gtex/heaps_null.pdf}
    \end{minipage}
\caption{Heaps' law; TCGA(left) and GTEx (right)}
\label{fig:structure/heaps_null}
\end{figure}
Another way to see this is looking at the histograms of the number of different genes expressed, actually the distribution of the~\ref{fig:structure/heaps_null} y axis. Figure~\ref{fig:structure/diffwordsDistr_null} shows that these distributions are completely different if one looks at the data and at the samples.
\begin{figure}[htb!]
\begin{minipage}{0.5\textwidth}
    \centering
    \includegraphics[width=0.95\linewidth]{pictures/structure/tcga/diffwordsDistr_null.pdf}
    \end{minipage}
\hspace{2mm}
\begin{minipage}{0.5\textwidth}
    \centering
    \includegraphics[width=0.95\linewidth]{pictures/structure/gtex/diffwordsDistr_null.pdf}
    \end{minipage}
\caption{Occurrence distributions; TCGA(left) and GTEx (right)}
\label{fig:structure/diffwordsDistr_null}
\end{figure}

%%tissue differentation
%%tissue separation
\section{Statistical laws differentiate by tissue}
Observing the GTEx dataset of healthy samples it is possible to study how it is possible to see the tissue differentiation and how to study tissues' differences,~\cite{mele2014} suggests the approach.

First of all could be interesting to study which is the fraction of transcript that can be explained by a certain number of genes.
One can reduce the realisations to the ones that share the tissue. Than one estimates the average per each component (gene), at this point one has the average abundance of each gene in a tissue, dividing by the sum of all the components it is possible to obtain the fraction of the total counts in the tissue due to each gene. Sorting from greater to smaller and integrating (cumulative summing) one have the fraction of transcript due to $1, 2, 3\dots$ genes. This is plot in~\ref{fig:structure/gtex/fraction_of_trascriptome}. 
\begin{figure}[htb!]
  \centering
  \includegraphics[width=0.9\linewidth]{pictures/structure/gtex/fraction_of_trascriptome.pdf}
  \caption{The integral of the sorted abundances for each tissue}
  \label{fig:structure/gtex/fraction_of_trascriptome}
\end{figure}
Here, if a curve is steep it means that a few genes' counts represent a great fraction of the total. If a curve is smooth it means that many genes are necessary to describe the whole trascriptome for that particular tissue.
This analisys shows that different tissues have a different complexity in terms of the number of genes necessary to build the trascriptome (in average).
In figure~\ref{fig:structure/gtex/fraction_of_trascriptome_Brain} the same analisys is done for the sub-tissues of Brain, also this sub-type separate by tissue.
\begin{figure}[htb!]
  \centering
  \includegraphics[width=0.9\linewidth]{pictures/structure/gtex/fraction_of_trascriptome_Brain.pdf}
  \caption{The integral of the sorted abundances for sub-types of Brain. This is done using TPM to avoid biases due to gene lengths. Blood is plotted for reference.}
  \label{fig:structure/gtex/fraction_of_trascriptome_Brain}
\end{figure}

Coming back to the Zipf's law~\ref{eq:zipf}, it is now obvious that~\ref{fig:structure/gtex/fraction_of_trascriptome} represents nothing but the integral of the Zipf's law. So estimating the Zipf looking at a tissue a time, it is evident that each tissue has its particular slope. The steeper the Zipf the simplest is the tissue: the transcript can be described with a few genes. In figure~\ref{fig:structure/gtex/zipf_tissue} the tissue with an extreme behaviour.
\begin{figure}[htb!]
  \centering
  \includegraphics[width=0.6\linewidth]{pictures/structure/gtex/zipf_tissue.pdf}
  \caption{The integral of the sorted abundances for each tissue}
  \label{fig:structure/gtex/zipf_tissue}
\end{figure}

The point where the~\ref{fig:structure/gtex/fraction_of_trascriptome} reaches $1$ corresponds to the total number of genes expressed, the remaining ones have a $0$ expression and do not contribute to the transcript. This can be visualised again with the Heaps' law. In figure~\ref{fig:structure/gtex/heaps_tissue} it is evident that there is some kind of tissue differentiation even when looking at the Heaps' law.
\begin{figure}[htb!]
  \centering
  \includegraphics[width=0.6\linewidth]{pictures/structure/gtex/heaps_tissue.pdf}
  \caption{The integral of the sorted abundances for each tissue}
  \label{fig:structure/gtex/heaps_tissue}
\end{figure}

\begin{figure}[htb!]
  \centering
  \includegraphics[width=0.6\linewidth]{pictures/structure/gtex/heaps_tissue_disease.pdf}
  \caption{The integral of the sorted abundances for each tissue}
  \label{fig:structure/gtex/heaps_tissue_disease}
\end{figure}

All these analysis suggest that there must be a sort of hidden structure in the data that is somehow related with the tissue each sample comes from. In particular there are many different Zipf's laws hidden behind the data and each sample is build looking at one of these a time. Also given two samples with a similar size, it happens that the number of genes necessary to build that realisation is not always the same (shown by Heaps' law) and it is somehow related to the tissue of the sample.


In conclusion, some interesting laws were found