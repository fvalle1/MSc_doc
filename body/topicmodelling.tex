\chapter{Topic model}\label{ch:topicmodelling}
Inspired by~\cite{gerlach2018network} is investigated the possibility to use a so called \textit{Topic model} to cluster genes and samples.
This is a complex approach to stochastic bloc models~\cite{Holland1983}

Topic models see a graph $G$ as generated by a set of latent
variables $\theta$ ($P(G|\theta)$) and try to reconstruct it maximising a posterior probability $P(\theta|G)$.
Documents are described as a mixture of topics $P(t|d)$, topics are nothing but
a mixture of words $P(w|t)$.

Referring to data used in this project documents will be \textbf{cases}, words will be \textbf{genes} and topics became
\textbf{cancer types}.

The main goal would be to reproduce cancer's classification and find genes
cancer's markers as words that are specific for a topic.

\section{Metrics}
As~\cite{rosenberg2007v} given $C$ and $K$

\begin{equation}\label{eq:homogeneity}
    h=1-\frac{h(C|K)}{H(C)}
\end{equation}

\begin{equation}\label{eq:completness}
    c=1-\frac{h(K|C)}{H(K)}
\end{equation}

\begin{equation}\label{eq:mutualinformation}
    MI=2\frac{h c}{h + c}
\end{equation}

\section{Run on GTEx}

\section{Run on TCGA}

\setion{Mixed runs}

\draft{hierarchical clustering}
\draft{This is better than LDA because...}