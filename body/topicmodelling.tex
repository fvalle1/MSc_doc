\chapter{Topic modelling}\label{ch:topicmodelling}
Once extensively analysed the structure of the dataset, the goal becomes to develop a machine learning method to learn the hidden structure of the data.

%%intro
Remembering that in chapter~\ref{ch:structure} it emerged some kind of structure behind data, where each tissue seemed to be sampled by a different power law, a topic modelling approach is here proposed.

The idea is that behind data there are hidden variables that describe the relationship between the genes and the samples. Let's call these variables topics.
Firstly it is necessary to build a bipartite network of genes and samples, then nodes are linked considering the gene expression value in the dataset.
\begin{figure}[htb!]
    \centering
    \includegraphics[width=0.7\linewidth]{pictures/topic/bipartite.pdf}
    \caption{An example of a bipartite network. Samples are on the left, genes are on the right. Each link is weighted by gene expression value. On the left side, all nodes of the same colour are clusters of samples. On the right side, all nodes with the same colour are set of genes, also known as topics.\\ 
    Blue lines represent the cluster structure, each blue square is a set of nodes, lines delineate the hierarchical structure.\\
    It is clear in the middle the network separation in genes and samples.}
    \label{fig:topic/bipartite}
\end{figure}

The output of this kind of models consists of sets of genes, the topics, with a probability distribution $P(\text{gene} | \text{topic})$ and probability distributions of these topics inside each sample $P(\text{topic} | \text{sample})$, together they give the relationship between a \textit{sample} and a \textit{gene}.

In this work it is used an innovative and recent approach to topic model, the algorithm was presented by~\cite{gerlach2018network} and well explained in details in~\cite{Peixoto2017}. This model is a sort of stochastic block models~\cite{Holland1983}. 
Topic modelling has being developed and studied to approach linguistics problems, so this algorithm was developed considering words and books in input, links represents the abundance of a word in a book. In chapter~\ref{ch:structure} was evident that there are many similarities between data considered in this work and linguistics' corpora. Referring to data used in this project \textbf{samples} will be the documents and \textbf{genes} will be the words.
It is expected that topics represent some properties of the system due to the gene expression distribution in samples.

The ultimate goal would be to be able to separate healthy and diseased samples then separate and find well-known tumour types, then extend the actual knowledge and retrieve the tumour sub-types.

One of the advantages of this particular algorithm is that it is hierarchical, so it applies community detection at different layers. So the output has got different resolutions and number of clusters found at each layer. One extreme layer is the one which separates genes ($\simeq$ words) and samples ($\simeq$ samples), by definition; in other layers it is possible to have few big clusters until the other extreme were the number of clusters is comparable with the number of nodes.
\begin{figure}[htb!]
  \centering
  \includegraphics[width=0.6\linewidth]{pictures/topic/peixioto_hierarchic.jpg}
  \caption{Example of a hierarchical structure. At $l=0$ the number of cluster is comparable with the number of nodes, is the situation with many small clusters.}
  \label{fig:topic_peixioto_hierarchic}
\end{figure}

What the algorithm does is to run a sort of Monte Carlo simulation and find the best partition of the data.
The probability that the hidden variables $\theta$ describe the data $G$ $P(\theta | G)$ can be written as a likelihood times a prior probability as 
\[P(\theta|G)=\frac{P(G|\theta)\overbrace{P(\theta)}^{prior}}{\underbrace{P(G)}_{\sum_{\theta}P(G|\theta)P(\theta)}}.\]
It is possible to define a description length
\[
\Sigma=-lnP(G|\theta)-lnP(\theta),
\]
so that $P(\theta | G)\propto e^{-\Sigma}$.
Moreover, the likelihood $P(G | \theta)$, can be written as $\frac{1}{\Omega}$ where $\Omega$ is the number of possibles realizations given $\theta$. This can be represented as a microcanonical ensemble with entropy $S=Ln\left(\Omega\right)$. According to~\cite{peixoto2017nonparametric} entropy $S$ can be written as
\[
S=\frac{1}{2}\Sigma_{r,s} n_r n_s H\left(\frac{e_{rs}}{n_rn_s}\right),
\]
where $n_r$ is the number of nodes in block $r$, $e_{rs}$ the number of links between nodes of group $r$ and group $s$ and $H$ is the Shannon entropy $H(x)=xLog_2(x)+(1-x)Log_2(1-x)$. Note that $S$ is minimal if $\frac{e_{rs}}{n_rn_s}$ is close to zero, $r$ and $s$ are two completely separated blocks or if it is close to $1$, $r$ and $s$ are groups with many connections; this allows to find groups with nodes very disconnected or topic and clusters with a lot of connections. Note that the description length depends on the entropy:
\[
\Sigma=S-lnP(\theta),
\]
The algorithm tries to minimize $S$, so that $\Sigma$ is minimized, so $e^{-\Sigma}$ is maximized, but this is $P(\theta | G)$ that is the required probability to maximize.

The Monte Carlo works in a few steps:
\begin{itemize}
 \item a node $i$ is chosen
 \item the group of $i$ is called $r$
  \item a node $j$ is chosen from $i$'s neighbours, the group of $j$ is called $t$
  \item a random group $s$ is selected
  \item move of node $i$ to group $s$ is accepted with probability $P(r\to s|t)=\frac{e_{ts}+\epsilon}{e_t+\epsilon B}$
  \item if $s$ is not accepted, a random edge $e$ is chosen from group $t$ and node $i$ is assigned to the endpoint of $e$ which is not in $t$
\end{itemize}
in figure~\ref{fig:topic_peixioto_move} an example of these steps.
\begin{figure}[htb!]
  \centering
  \includegraphics[width=0.9\linewidth]{pictures/topic/peixioto_move.jpg}
  \caption{Left: Local neighbourhood of node $i$ belonging to block $r$, and a randomly chosen neighbour $j$ \
  belonging to block $t$. \
  Right: Block multi graph, indicating the number of edges between blocks, represented as the edge thickness. \
  In this example, the attempted move $bi \to s$ is made with a larger probability than either $bi \to u$\
   or $bi \to r$ (no movement), since $e_{ts}>e_{tu}$ and $e_{ts}>e_{tr}$.}
  \label{fig:topic_peixioto_move}
\end{figure}
Once the model run it is possible to estimate the probability distribution of words inside a topic
\[P(w|t_w)=\frac{\text{\# of edges on $w$ to $t_w$}}{\text{\# of edges on $t_w$}}\]
and the topic distribution inside a document
\[P(t_w|d)=\frac{\text{\# of edges on $d$ from $t_w$}}{\text{\# of edges on $d$}}\]
In the case of overlapping partitions, the presence of a word in a topic is not trivial and can be estimated as 
\[P(t_w|w)=\frac{\text{\# of edges on $w$ to $t_w$}}{\text{\# of edges on $w$}}\]
or the presence of a document in a cluster
\[P(t_d|d)=\frac{\text{\# of edges on $d$ to $t_d$}}{\text{\# of edges on $d$}}\]

See appendix~\ref{app:hsbm} for detailed analysis of the maths behind the algorithm and \url{https://cloud.docker.com/repository/docker/fvalle01/hsbm} for the extension of~\cite{gerlach2018network} to non linguistics component systems datasets.


%%metrics
\section{Test the model with metrics and benchmarks}
Before running topic modelling, it is useful to define some metrics to test and benchmark the model. In particular the model searches sets on the two sides of the network: the one containing samples and the one containing genes. Samples are extracted from datasets where much metadata are available, some of these metadata labels will be used to benchmark the model. To study genes, enrichment test are instead necessary.

Looking at the samples side of the network, the outputs are sets of samples, let's call these clusters. One can state that the model works if all, or at least the majority, of samples in the same cluster share some label. Here the sample primary site is considered as the main label.

Note that this work's model is a non-supervised one, but a ground truth is available from metadata. So every sample has a certain probability to have a certain property (the true tissue label), let's call this $P(C)$ and a certain probability of being in a cluster (model's output), let's call this $P(K)$.
It is possible to define, for instance, the homogeneity
\begin{equation}\label{eq:homogeneity}
    h=1-\frac{H(C|K)}{H(C)}
\end{equation}
defining the entropy
\begin{equation}\label{eq:hck}
    H(C|K)=\sum_{c\in \mathrm{labels},\\ k \in \mathrm{clusters}}\frac{n_{c k}}{N}Log\left(\frac{n_{c k}}{n_k}\right)
\end{equation}
where $n_{c k}$ is the number of nodes of type $c$ in cluster $k$, $N$ the number of nodes and $n_k$ the number of nodes in cluster $k$. It is evident that if all nodes inside cluster $k$ are of the same type $c$ $n_{c k}=n_{k}$, $H(C|K)=0$ and $h=1$, it is actually a full homogeneous situation.

Another quantity can be defined: the so-called completeness
\begin{equation}\label{eq:completness}
    c=1-\frac{H(K|C)}{H(K)},
\end{equation}
$H(K|C)$ is defined in the same way as~\ref{eq:hck}. Completeness measures how well nodes of the same type are distributed in the same cluster.

Ideally one wants a method which output is both homogeneous and complete. So it is possible to define the V-measure as the harmonic average of the two:
\begin{equation}\label{eq:mutualinformation}
    \mathrm{V-measure}=2\frac{h c}{h + c},
\end{equation}
which is actually the normalized mutual information between $P(C)$ and $P(K)$~\cite{rosenberg2007v}. Please refer to appendix~\ref{app:vmeasure} for the detailed maths. In figure~\ref{fig:topic/cartoon/hc_table} a simplified example of the homogeneity and completenesses ideas. ~\cite{Shi} proposed a similar metric to compare topic modelling algorithms' performances.
\begin{table}[htb!]
	\centering
	\begin{tabular}{|c|c|c|}
		\hline
		&Homogeneous & Not homogeneous\\ \hline
		\rotatebox[origin=l]{90}{Complete}&    \includegraphics[width=0.2\textwidth]{pictures/topic/cartoon/cartoon_hc.pdf}&\includegraphics[width=0.2\textwidth]{pictures/topic/cartoon/cartoon_c.pdf}  \\ \hline
		\rotatebox{90}{Not complete}&   \includegraphics[width=0.2\textwidth]{pictures/topic/cartoon/cartoon_h.pdf}&\includegraphics[width=0.2\textwidth]{pictures/topic/cartoon/cartoon.pdf} \\ \hline
	\end{tabular}
	\caption{Examples of homogeneity and completeness. Homogeneous clusters contain all nodes with the same label. A label is complete if it is fully represented by a single cluster. In this image some extreme examples of these definitions.}
	\label{tab:topic/cartoon/hc_table}
\end{table}
			
In figure~\ref{fig:topic/metric_scores_primarysite} an example of the V-measure score estimated at the different layers of the hierarchy; note that the number of clusters increases going deeper in the hierarchy. In the same figure homogeneity and completeness are reported, note that with few clusters the situation is more complete, but when the number of clusters increases completeness goes down and homogeneity increases. This happens because if a cluster is small it is easier to fulfil it with similar objects. On the other side if one has few clusters it is easier to complete them putting similar objects in the same cluster.
\begin{figure}[htb!]
    \centering
    \includegraphics[width=0.8\linewidth]{pictures/topic/gtex/oversigma_10tissue/metric_scores_primarysite.pdf}
    \caption{Score across hierarchy. The V-measure or normalized mutual information MI is the harmonic average between homogeneity and completeness.}
    \label{fig:topic/metric_scores_primarysite}
\end{figure}

In order to validate the model, it will be compared to more standard approaches such as standard hierarchical clustering and a classical topic model approach using Latent Dirichlet Allocation. 
\FloatBarrier

%%comparables algorithms
\subsection{LDA}
\draft{commenti vari}

As in ~\cite{Zhou2016}
\begin{equation}\label{eq:lda}
  P(w, z,\beta, \theta| \alpha, \eta)=\prod_n^{N_d} P(w|z,\beta)P(z|\theta)\prod_k^KP(\beta|\eta)\prod_d^N P(\theta | \alpha)
\end{equation}

\begin{figure}
  \centering
  \includegraphics[width=0.5\linewidth]{pictures/topic/LDA.jpeg}
  \label{fig:LDA}
  \caption{LAD scheme}
\end{figure}
where
\begin{itemize}
  \item $N$ number of documents
  \item $K$ number of topics
  \item $w$ words
  \item $N_d$ number of words in document d
  \item $\eta$ and $\alpha$ are parameters of the model
\end{itemize}
in~\ref{eq:lda} $P(\theta | \alpha)$ and $P(\beta|\eta)$ are Dirichlet distributions the outputs are the topic distribution in documents $P(z|d)$ and the word distribution in topics $P(w|z)$

\subsection{Hierarchical clustering}

\draft{da scrivere}

\draft{hierarchical clustering}
Hierarchical clustering is a general family of clustering algorithms that build nested clusters by merging or splitting them successively. This hierarchy of clusters is represented as a tree (or dendrogram). The root of the tree is the unique cluster that gathers all the samples, the leaves being the clusters with only one sample. See the Wikipedia page for more details.

The AgglomerativeClustering object performs a hierarchical clustering using a bottom up approach: each observation starts in its own cluster, and clusters are successively merged together. The linkage criteria determines the metric used for the merge strategy:

Ward minimizes the sum of squared differences within all clusters. It is a variance-minimizing approach and in this sense is similar to the k-means objective function but tackled with an agglomerative hierarchical approach.
Maximum or complete linkage minimizes the maximum distance between observations of pairs of clusters.
Average linkage minimizes the average of the distances between all observations of pairs of clusters.
Single linkage minimizes the distance between the closest observations of pairs of clusters.
AgglomerativeClustering can also scale to large number of samples when it is used jointly with a connectivity matrix, but is computationally expensive when no connectivity constraints are added between samples: it considers at each step all the possible merges.


\begin{lstlisting}[style=mypython]
from sklearn.cluster import AgglomerativeClustering
AgglomerativeClustering(
    affinity='euclidean',
    compute_full_tree='auto',
    linkage='ward',
    n_clusters=x,
    )
\end{lstlisting}



%%preprocess
\section{Pre-process}
To make the algorithm faster, it could be useful to do a pre-processing of the data.
Different approaches were tested, all of them involving the quantities defined in~\ref{ch:structure}. The goal is to identify components which are able to best separate the realizations. 
\paragraph{Low occurrence genes} were selected firstly to approach topic modelling. A $0.5$ threshold was set on occurrence. This method selects genes that appears (have expression greater than zero) only in less than half samples. This approach has some limitations, for instance it doesn't consider genes that appear everywhere (with occurrence $\simeq 1$) but changes their behaviour across realisations.

\paragraph{tf-idf (term frequency–inverse document frequency)} should help. This approach doesn't take in account original expression values $n_{ij}$, but a transformed version
\[
n^{new}_{ij}=\frac{n_{i j}}{M_j}\times \left(1-Log\left(o_i\right)\right)
\] which increases the importance of components with small occurrence $o_i$. This approach doesn't actually select components, which is still an issue.

\paragraph{Highly variable} genes can be selected. This is done using the $CV^2$ analysis done in chapter~\ref{ch:scalinglaws}.
\begin{figure}[htb!]
    \centering
    \includegraphics[width=0.8\linewidth]{pictures/topic/cvmean_oversigma.png}
    \caption{Highly variable genes}
    \label{fig:topic/cvmean_oversigma}
\end{figure}
Plotting the coefficient of variation versus the mean for each component reveals which components have higher variance with respect to components which, on average, have a similar behaviour. Binned averages and variances were estimated, and only genes with a $CV^2$ over a $\sigma$ greater than the bin's mean were considered. This method seems to select useful genes even if the binned average bound is quite noisy.

\paragraph{Distance from boundaries} can be a similar and alternative method to select highly variable genes. In this case the bound is smooth and well defined.
\begin{figure}[htb!]
    \centering
    \includegraphics[width=0.8\linewidth]{pictures/topic/cvmean_oversampling.png}
    \caption{Genes distant from the boundaries}
    \label{fig:topic/cvmean_oversampling}
\end{figure}
The distribution as discussed in~\ref{ch:scalinglaws} have a Poisson-like and a Taylor-like boundaries. So can be considered only components that are the most distant from these boundaries. Moreover this boundaries can be found with a simple null model, as shown in figure~\ref{fig:scalinglaws/gtex/cvmean_loglog_sampling} the sampling model defines the lower bound of the data.

The last two approaches are the ones which lead to better results, in the following sections gene selection was done by getting only highly variables genes.

\clearpage
\section{Run}
%%gtex
\subsection{Run on GTEx}
Once the model was tuned and adapted to RNA-Sequencing data, it was run on a subset of the GTEx dataset. A subset was chosen randomly in order to reduce the computing time needed. The analysis hereby described took about 2 days to be run on a 16 core CPU, 100GB memory facility. The great amount of memory is needed to temporary store the network configuration at each step of the Monte Carlo simulation.

First of all to rapidly have an information about the interest of the oncoming result the metric above described were considered. In figure~\ref{fig:topic/gtex/oversigma_10tissue/metric_scores} it is represented the V-measure score versus the number of clusters found at different layers.
\begin{figure}[htb!]
    \centering
    \includegraphics[width=0.9\linewidth]{pictures/topic/gtex/oversigma_10tissue/metric_scores.pdf}
    \caption{Scores across hierarchy. The primary site and secondary site labels are compared.}
    \label{fig:topic/gtex/oversigma_10tissue/metric_scores}
\end{figure}
The result is quite good, the maximum score is over $0.8$. Considering that, for example,~\cite{Farver2018} obtained a similar score considering only homogeneity, this is a quite good result. A second interesting fact is that both the tissue label (primary site) and the sub tissue label (secondary site) obtain such a good score, moreover the the secondary site score's peak is at an higher number of clusters coherently with the fact that there is a greater number of sub tissue labels.
This score can be useful to define at which level of the hierarchy the consequent analysis should be made.

In figure~\ref{fig:topic/gtex/oversigma_10tissue/bipartite_rebuild} the relation between the clusters at different layers it is evident.
\begin{figure}[htb!]
    \centering
    \includegraphics[width=0.8\linewidth]{pictures/topic/gtex/oversigma_10tissue/bipartite_rebuild.pdf}
    \caption{Hierarchy of the files' nodes.}
    \label{fig:topic/gtex/oversigma_10tissue/bipartite_rebuild}
\end{figure}

\clearpage

In figure~\ref{fig:topic/gtex/oversigma_10tissue/clustercomposition_l3_primary_site} each column is a cluster and each color is a tissue of the dataset. It is evident that the majority of the tissue are identified: the first, second, fifth, sixth, eighth and tenth columns are fully and uniformly colored of the same color. These corresponds to an identification of brain, blood, lung, testis and bladder.
\begin{figure}[htb!]
    \centering
    \includegraphics[width=0.9\linewidth]{pictures/topic/gtex/oversigma_10tissue/clustercomposition_l3_primary_site.pdf}
    \caption{Clusters composition at level of the hierarchy with higher score. Each column is a cluster, each color is a label.}
    \label{fig:topic/gtex/oversigma_10tissue/clustercomposition_l3_primary_site}
\end{figure}
A normalised representation of the same clusters the result is still quite interesting and the homogeneity of the clusters is more evident.
\begin{figure}[htb!]
    \centering
    \includegraphics[width=0.9\linewidth]{pictures/topic/gtex/oversigma_10tissue/fraction_clustercomposition_l3_primary_site.pdf}
    \caption{Normalised composition of clusters.}
    \label{fig:topic/gtex/oversigma_10tissue/fraction_clustercomposition_l3_primary_site}
\end{figure}
Going deeper in the hierarchy and looking at a layer with more cluster the result, shown in figure~\ref{fig:topic/gtex/oversigma_10tissue/fraction_clustercomposition_l2_primary_site}, demonstrates that at this point all the the tissues are separated and each cluster is fully of nodes sharing the same tissue.
\begin{figure}[htb!]
    \centering
    \includegraphics[width=0.9\linewidth]{pictures/topic/gtex/oversigma_10tissue/fraction_clustercomposition_l2_primary_site.pdf}
    \caption{Normalised composition of clusters at a deeper level.}
    \label{fig:topic/gtex/oversigma_10tissue/fraction_clustercomposition_l2_primary_site}
\end{figure}
Even looking at sub-tissues the results is quite good. It is not always easy to separate all the sub-parts of brain, nevertheless the cerebellum is well identified (column 13) and blood is distinguished in whole blood (columns 1-4) and lymphocytes (column 10).
\begin{figure}[htb!]
    \centering
    \includegraphics[width=0.9\linewidth]{pictures/topic/gtex/oversigma_10tissue/fraction_clustercomposition_l2_secondary_site.pdf}
    \caption{Normalised composition of clusters with respect to the secondary site subtissue labels.}
    \label{fig:topic/gtex/oversigma_10tissue/fraction_clustercomposition_l2_secondary_site}
\end{figure}

\subsection{Shuffling}
A null model of cluster composition is necessary In order to be able to state that a result is better than expected. This was done by doing the same analysis but reshuffling the labels of the nodes. Doing so the number of clusters and the cluster sizes are maintained. In figure~\ref{fig:topic/gtex/oversigma_10tissue/shuffledclustercomposition_l3_primary_site} an example of clustering with random labels, it is evident that all clusters have similar and homogeneous composition. Note that not every tissue has the same number of samples, so for example blood is more represented than other tissues.
\begin{figure}[htb!]
	\centering
	\includegraphics[width=0.8\linewidth]{pictures/topic/gtex/oversigma_10tissue/shuffledclustercomposition_l3_primary_site}
	\caption{Example of visualization of clusters with reshuffled labels.}
	\label{fig:topic/gtex/oversigma_10tissue/shuffledclustercomposition_l3_primary_site}
\end{figure}

All the results described in the previous pictures are quite qualitative. To have a more objective and mathematical measure of the success of the algorithm it is possible to measure the fraction of the most representative label in each cluster $k$
\[
max_{l\in labels}\left(\frac{n_{l k}}{n_k}\right)
\]
with $n_{l k}$ is the numbers of nodes labeled $l$ in cluster $k$ and $n_k$ is the number of nodes in cluster $k$. This is represented in figure~\ref{fig:gtex/oversigma_10tissue/shuffledcluster_maximum_l2_primary_site} for the level where the V-measure is maximized (best results are expected here). In figure~\ref{fig:gtex/oversigma_10tissue/shuffledcluster_maximum_l2_primary_site} on the left is shown the most representative label fraction versus for each cluster, on the right the histograms of the same quantity. It is evident that models' clusters are very homogeneous with the majority of cluster with almost $100\%$ of the same tissue. It is also clear that reshuffling the labels the result is very different and so the models behave better than expected. 
\begin{figure}[htb!]
    \centering
    \includegraphics[width=0.9\linewidth]{pictures/topic/gtex/oversigma_10tissue/shuffledcluster_maximum_l2_primary_site.pdf}
    \caption{Most representative label versus cluster size.}
    \label{fig:gtex/oversigma_10tissue/shuffledcluster_maximum_l2_primary_site}
\end{figure}
In figure~\ref{fig:topic/gtex/oversigma_10tissue/shuffledcluster_maximum*} the same analysis is done for every level of the hierarchy. It is interesting to notice that at deeper levels (upper left in the figure) the random reshuffling and the real labels have the same behavior. This is due to the fact that at this level clusters are very small and so it is easier to pick up nodes with the same level (in the extreme case of a cluster with size 1 it is always full with the same label). This shows that the deeper level it is not interesting, results are exactly the same with random labels;  moreover the reshuffling null model it is good to show up eventual biases due to small cluster sizes.
\begin{figure}[htb!]
    \centering
    \begin{minipage}{0.45\textwidth}
    \includegraphics[width=0.9\linewidth]{pictures/topic/gtex/oversigma_10tissue/shuffledcluster_maximum_l0_primary_site.pdf}
    \end{minipage}
    \hspace{3mm}
    \begin{minipage}{0.45\textwidth}
    \includegraphics[width=0.9\linewidth]{pictures/topic/gtex/oversigma_10tissue/shuffledcluster_maximum_l1_primary_site.pdf}
    \end{minipage}
    \\
    \begin{minipage}{0.45\textwidth}
    \includegraphics[width=0.9\linewidth]{pictures/topic/gtex/oversigma_10tissue/shuffledcluster_maximum_l2_primary_site.pdf}
    \end{minipage}
    \hspace{3mm}
    \begin{minipage}{0.45\textwidth}
    \includegraphics[width=0.9\linewidth]{pictures/topic/gtex/oversigma_10tissue/shuffledcluster_maximum_l3_primary_site.pdf}
    \end{minipage}
    \caption{Fraction of the most representative label in all clusters for different levels of the hierarchy. From upper left the deeper layer than down right the upper one.}
    \label{fig:topic/gtex/oversigma_10tissue/shuffledcluster_maximum*}
\end{figure}

A similar analysis can be made considering not just the number of the cluster but the cluster size, this is shown in figure~\ref{fig:topic/gtex/oversigma_10tissue/shuffledclusterhomosize_l3_primary_site}. It is interesting to notice that the shuffle null model and the real labels clusters are evidently different, so there must be some kind of signal. It is clear that the model is able to output big clusters full with the same label.
\begin{figure}[htb!]
	\centering
	\includegraphics[width=0.9\linewidth]{pictures/topic/gtex/oversigma_10tissue/shuffledclusterhomosize_l3_primary_site.pdf}
	\caption{Fraction of most representative label versus cluster size.}
	\label{fig:topic/gtex/oversigma_10tissue/shuffledclusterhomosize_l3_primary_site}
\end{figure}
In figure~\ref{fig:topic/gtex/oversigma_10tissue/shuffledclusterhomosize_l*} the same analysis for  all the levels of the hierarchy. It is interesting to see how going up in the hierarchy the two signals become different, as shown before the deeper layer (upper left in the image) is not different from null model and so it is not interesting.
\begin{figure}[htb!]
	\centering
	\begin{minipage}{0.45\textwidth}
		\includegraphics[width=0.9\linewidth]{pictures/topic/gtex/oversigma_10tissue/shuffledclusterhomosize_l0_primary_site.pdf}
	\end{minipage}
	\hspace{3mm}
	\begin{minipage}{0.45\textwidth}
		\includegraphics[width=0.9\linewidth]{pictures/topic/gtex/oversigma_10tissue/shuffledclusterhomosize_l1_primary_site.pdf}
	\end{minipage}
	\\
	\begin{minipage}{0.45\textwidth}
		\includegraphics[width=0.9\linewidth]{pictures/topic/gtex/oversigma_10tissue/shuffledclusterhomosize_l2_primary_site.pdf}
	\end{minipage}
	\hspace{3mm}
	\begin{minipage}{0.45\textwidth}
		\includegraphics[width=0.9\linewidth]{pictures/topic/gtex/oversigma_10tissue/shuffledclusterhomosize_l3_primary_site.pdf}
	\end{minipage}
	\caption{Fraction of most representative label versus cluster size across the hierarchy. From upper left the deeper layer than down right the upper one.}
	\label{fig:topic/gtex/oversigma_10tissue/shuffledclusterhomosize_l*}
\end{figure}


At this point to deepen investigate the structure of the clusters it can be interesting to study how many labels are present in each cluster. In fact the fraction of most represented label defined above carries no information of what happens to the non most representative labels. For example if one cluster is composed of $80\%$ by label \textbf{A} and $20\%$ by label \textbf{B} and another cluster is composed $80\%$ by label \textbf{A}, $10\%$ by label \textbf{B} and $10\%$ by label \textbf{C} they have both fraction of maximum representative label $80\%$ but the second in this example is more heterogeneous. Counting the number of different labels in each cluster can reveal this sort of effects. In figure~\ref{fig:topic/gtex/oversigma_10tissue/shuffledcluster_shuffle_label_size_l3_primary_site} it is represented the number of different labels versus cluster size. It is evident that the reshuffling case is quite different from the real one, almost every cluster in the null model has got every label. It is interesting to notice that the model outputs even big cluster with one label.
\begin{figure}[htb!]
    \centering
    \includegraphics[width=0.9\linewidth]{pictures/topic/gtex/oversigma_10tissue/shuffledcluster_shuffle_label_size_l3_primary_site.pdf}
    \caption{Number of different labels in each cluster versus cluster size.}
    \label{fig:topic/gtex/oversigma_10tissue/shuffledcluster_shuffle_label_size_l3_primary_site}
\end{figure}
In figure~\ref{fig:topic/gtex/oversigma_10tissue/shuffledcluster_shuffle_label_size_l*} the same analysis for all the layer of the hierarchy. Even here the deeper level does not differ from the null model. Nevertheless in layers with higher V-measure score there is a strong signal that the reshuffling model is quite different from the real labels output.
\begin{figure}[htb!]
    \centering
    \begin{minipage}{0.45\textwidth}
    \includegraphics[width=0.9\linewidth]{pictures/topic/gtex/oversigma_10tissue/shuffledcluster_shuffle_label_size_l0_primary_site.pdf}
    \end{minipage}
    \hspace{3mm}
    \begin{minipage}{0.45\textwidth}
    \includegraphics[width=0.9\linewidth]{pictures/topic/gtex/oversigma_10tissue/shuffledcluster_shuffle_label_size_l1_primary_site.pdf}
    \end{minipage}
    \\
    \begin{minipage}{0.45\textwidth}
    \includegraphics[width=0.9\linewidth]{pictures/topic/gtex/oversigma_10tissue/shuffledcluster_shuffle_label_size_l2_primary_site.pdf}
    \end{minipage}
    \hspace{3mm}
    \begin{minipage}{0.45\textwidth}
    \includegraphics[width=0.9\linewidth]{pictures/topic/gtex/oversigma_10tissue/shuffledcluster_shuffle_label_size_l3_primary_site.pdf}
    \end{minipage}
\label{fig:topic/gtex/oversigma_10tissue/shuffledcluster_shuffle_label_size_l*}
\caption{Number of different labels in each cluster versus cluster size. From upper left the deeper layer than down right the upper one.}
\end{figure}

Having constructed the null model it is possible estimate the V-measure score also for the null model. The results are reported in figure~\ref{fig:topic/gtex/oversigma_10tissue/metric_scores_shuffle}. 
Moreover remembering the V-measure or normalized mutual information defined in~\ref{eq:mutualinformation} it is possible to estimate a mixed score which considers the homogeneity of primary site and the completeness of secondary site, doing so the score goes up if going deeper in the hierarchy the model makes more cluster with the same tissue but separates sub tissues. In fact it is not a big deal if one loses completeness regarding tissues (the model separates one big cluster full with the same label into two small ones) but gain information at the next magnification information. This is clear if one look at the big blood cluster that in the next level of the hierarchy is separated into two clusters of blood, one of whole blood and one of lymphocytes. The result is that this mixed score is the highest one.
\begin{figure}[htb!]
    \centering
    \includegraphics[width=0.9\linewidth]{pictures/topic/gtex/oversigma_10tissue/metric_scores_shuffle.pdf}
    \caption{Scores across hierarchy. The scored is compared with some random labels. In blue the score for the primary site labels, in red for the secondary site labels, in yellow the shuffled labels, in green the mixed score with primary homogeneity and secondary completeness.}
    \label{fig:topic/gtex/oversigma_10tissue/metric_scores_shuffle}
\end{figure}

\clearpage
\subsection{Standard algorithms}
At this point was verified that the model has got interesting output, it reaches high scores and has got a strong signal against null model, at least at some levels of the hierarchy. It is now interesting to compare it with standard and well-known similar algorithms.
First of all a comparison is made with hierarchical clustering. This is done using the standard scipy~\cite{jones2014scipy} package, the metrics used was the euclidean one and the linkage method was set to Ward. This is quite fast, it needs a couple of minutes on a dual core, 8GB memory machine.
In figure~\ref{fig:topic/gtex/oversigma_10tissue/metric_scores_hier} the comparison between this scores, the hierarchical algorithm performs worse than hierarchical stochastic block model and as highly expected better than the random model.
\begin{figure}[htb!]
    \centering
    \includegraphics[width=0.9\linewidth]{pictures/topic/gtex/oversigma_10tissue/metric_scores_hier.pdf}
    \caption{Scores across hierarchy. The scored is compared with some random labels}
    \label{fig:topic/gtex/oversigma_10tissue/metric_scores_hier}
\end{figure}

Another very used and well-studied algorithm is Latent Dirichlet Allocation briefly described in~\ref{sec:lda}
\begin{figure}[htb!]
    \centering
    \includegraphics[width=0.9\linewidth]{pictures/topic/gtex/oversigma_10tissue/metric_scores_all.pdf}
    \caption{Scores across hierarchy. The scored is compared with some random labels}
    \label{fig:topic/gtex/oversigma_10tissue/metric_scores_all}
\end{figure}

\subsection{Topics}
\draft{Using gsea ~\cite{subramanian2005gene} as gene set ~\cite{Ardlie2015} enrichment test can be made from python \cite{Kuleshov2016}}

\begin{table}[htb!]
	\tiny
	\begin{center}
		\begin{tabular}{|c|c|c|}
			\hline
			Term & \multicolumn{1}{l|}{Adjusted P-value} & Gene\_set \\ \hline
			pancreas\_male\_60-69\_years & 1E-19 & GTEx\_Tissue\_Sample\_Gene\_Expression\_Profiles\_up \\ \hline
			pancreas\_female\_40-49\_years & 3E-19 & GTEx\_Tissue\_Sample\_Gene\_Expression\_Profiles\_up \\ \hline
			pancreas\_male\_40-49\_years & 5E-19 & GTEx\_Tissue\_Sample\_Gene\_Expression\_Profiles\_up \\ \hline
			pancreas\_male\_30-39\_years & 1E-18 & GTEx\_Tissue\_Sample\_Gene\_Expression\_Profiles\_up \\ \hline
			pancreas\_female\_20-29\_years & 1E-18 & GTEx\_Tissue\_Sample\_Gene\_Expression\_Profiles\_up \\ \hline
			pancreas\_male\_50-59\_years & 1E-18 & GTEx\_Tissue\_Sample\_Gene\_Expression\_Profiles\_up \\ \hline
			pancreas\_female\_30-39\_years & 1E-18 & GTEx\_Tissue\_Sample\_Gene\_Expression\_Profiles\_up \\ \hline
			pancreas\_male\_50-59\_years & 2E-18 & GTEx\_Tissue\_Sample\_Gene\_Expression\_Profiles\_up \\ \hline
			pancreas\_male\_40-49\_years & 2E-18 & GTEx\_Tissue\_Sample\_Gene\_Expression\_Profiles\_up \\ \hline
			pancreas\_male\_30-39\_years & 2E-18 & GTEx\_Tissue\_Sample\_Gene\_Expression\_Profiles\_up \\ \hline
			pancreas\_male\_50-59\_years & 2E-18 & GTEx\_Tissue\_Sample\_Gene\_Expression\_Profiles\_up \\ \hline
			pancreas\_female\_20-29\_years & 2E-18 & GTEx\_Tissue\_Sample\_Gene\_Expression\_Profiles\_up \\ \hline
			pancreas\_male\_40-49\_years & 3E-18 & GTEx\_Tissue\_Sample\_Gene\_Expression\_Profiles\_up \\ \hline
			pancreas\_female\_50-59\_years & 4E-18 & GTEx\_Tissue\_Sample\_Gene\_Expression\_Profiles\_up \\ \hline
			pancreas\_male\_50-59\_years & 4E-18 & GTEx\_Tissue\_Sample\_Gene\_Expression\_Profiles\_up \\ \hline
			pancreas\_male\_50-59\_years & 4E-18 & GTEx\_Tissue\_Sample\_Gene\_Expression\_Profiles\_up \\ \hline
			pancreas\_female\_60-69\_years & 5E-18 & GTEx\_Tissue\_Sample\_Gene\_Expression\_Profiles\_up \\ \hline
			pancreas\_female\_50-59\_years & 5E-18 & GTEx\_Tissue\_Sample\_Gene\_Expression\_Profiles\_up \\ \hline
			pancreas\_male\_50-59\_years & 5E-18 & GTEx\_Tissue\_Sample\_Gene\_Expression\_Profiles\_up \\ \hline
			pancreas\_male\_30-39\_years & 6E-18 & GTEx\_Tissue\_Sample\_Gene\_Expression\_Profiles\_up \\ \hline
		\end{tabular}
	\end{center}
	\caption{Enrichment}
	\label{topic/enrich/pancreas}
\end{table}

\begin{table}[htb!]
	\tiny
	\begin{center}
		\begin{tabular}{|c|c|c|}
			\hline
			Term & \multicolumn{1}{l|}{Adjusted P-value} & Gene\_set \\ \hline
			brain\_female\_40-49\_years & 6E-05 & GTEx\_Tissue\_Sample\_Gene\_Expression\_Profiles\_up \\ \hline
			brain\_male\_50-59\_years & 6E-05 & GTEx\_Tissue\_Sample\_Gene\_Expression\_Profiles\_up \\ \hline
			brain\_female\_60-69\_years & 6E-05 & GTEx\_Tissue\_Sample\_Gene\_Expression\_Profiles\_up \\ \hline
			brain\_female\_60-69\_years & 6E-05 & GTEx\_Tissue\_Sample\_Gene\_Expression\_Profiles\_up \\ \hline
			brain\_female\_60-69\_years & 6E-05 & GTEx\_Tissue\_Sample\_Gene\_Expression\_Profiles\_up \\ \hline
			brain\_female\_40-49\_years & 6E-05 & GTEx\_Tissue\_Sample\_Gene\_Expression\_Profiles\_up \\ \hline
			brain\_female\_40-49\_years & 6E-05 & GTEx\_Tissue\_Sample\_Gene\_Expression\_Profiles\_up \\ \hline
			brain\_female\_60-69\_years & 6E-05 & GTEx\_Tissue\_Sample\_Gene\_Expression\_Profiles\_up \\ \hline
			brain\_male\_60-69\_years & 6E-05 & GTEx\_Tissue\_Sample\_Gene\_Expression\_Profiles\_up \\ \hline
			brain\_male\_50-59\_years & 6E-05 & GTEx\_Tissue\_Sample\_Gene\_Expression\_Profiles\_up \\ \hline
			brain\_male\_50-59\_years & 6E-05 & GTEx\_Tissue\_Sample\_Gene\_Expression\_Profiles\_up \\ \hline
			brain\_male\_60-69\_years & 7E-05 & GTEx\_Tissue\_Sample\_Gene\_Expression\_Profiles\_up \\ \hline
			brain\_male\_50-59\_years & 7E-05 & GTEx\_Tissue\_Sample\_Gene\_Expression\_Profiles\_up \\ \hline
			brain\_male\_20-29\_years & 7E-05 & GTEx\_Tissue\_Sample\_Gene\_Expression\_Profiles\_up \\ \hline
			brain\_female\_60-69\_years & 8E-05 & GTEx\_Tissue\_Sample\_Gene\_Expression\_Profiles\_up \\ \hline
			brain\_female\_60-69\_years & 8E-05 & GTEx\_Tissue\_Sample\_Gene\_Expression\_Profiles\_up \\ \hline
			brain\_female\_60-69\_years & 1E-04 & GTEx\_Tissue\_Sample\_Gene\_Expression\_Profiles\_up \\ \hline
			brain\_female\_60-69\_years & 1E-04 & GTEx\_Tissue\_Sample\_Gene\_Expression\_Profiles\_up \\ \hline
			brain\_female\_60-69\_years & 1E-04 & GTEx\_Tissue\_Sample\_Gene\_Expression\_Profiles\_up \\ \hline
			brain\_male\_60-69\_years & 1E-04 & GTEx\_Tissue\_Sample\_Gene\_Expression\_Profiles\_up \\ \hline
		\end{tabular}
	\end{center}
	\caption{Enrichment}
	\label{topic/enrich/brain}
\end{table}

\begin{table}[htb!]
	\centering
	\tiny
	\begin{tabular}{|c|c|c|}
		\hline
		Term & \multicolumn{1}{l|}{Adjusted P-value} & Gene\_set \\ \hline
		blood\_male\_50-59\_years & 3E-23 & GTEx\_Tissue\_Sample\_Gene\_Expression\_Profiles\_up \\ \hline
		blood\_male\_50-59\_years & 3E-23 & GTEx\_Tissue\_Sample\_Gene\_Expression\_Profiles\_up \\ \hline
		blood\_male\_40-49\_years & 3E-21 & GTEx\_Tissue\_Sample\_Gene\_Expression\_Profiles\_up \\ \hline
		blood\_male\_60-69\_years & 9E-21 & GTEx\_Tissue\_Sample\_Gene\_Expression\_Profiles\_up \\ \hline
		blood\_male\_40-49\_years & 3E-20 & GTEx\_Tissue\_Sample\_Gene\_Expression\_Profiles\_up \\ \hline
		blood\_female\_60-69\_years & 4E-20 & GTEx\_Tissue\_Sample\_Gene\_Expression\_Profiles\_up \\ \hline
		blood\_male\_60-69\_years & 4E-20 & GTEx\_Tissue\_Sample\_Gene\_Expression\_Profiles\_up \\ \hline
		blood\_female\_50-59\_years & 5E-20 & GTEx\_Tissue\_Sample\_Gene\_Expression\_Profiles\_up \\ \hline
		blood\_female\_50-59\_years & 1E-19 & GTEx\_Tissue\_Sample\_Gene\_Expression\_Profiles\_up \\ \hline
		blood\_male\_60-69\_years & 1E-19 & GTEx\_Tissue\_Sample\_Gene\_Expression\_Profiles\_up \\ \hline
		blood\_male\_60-69\_years & 1E-19 & GTEx\_Tissue\_Sample\_Gene\_Expression\_Profiles\_up \\ \hline
		blood\_female\_60-69\_years & 1E-19 & GTEx\_Tissue\_Sample\_Gene\_Expression\_Profiles\_up \\ \hline
		blood\_male\_60-69\_years & 2E-19 & GTEx\_Tissue\_Sample\_Gene\_Expression\_Profiles\_up \\ \hline
		blood\_male\_50-59\_years & 2E-19 & GTEx\_Tissue\_Sample\_Gene\_Expression\_Profiles\_up \\ \hline
		blood\_female\_40-49\_years & 2E-19 & GTEx\_Tissue\_Sample\_Gene\_Expression\_Profiles\_up \\ \hline
		blood\_female\_40-49\_years & 2E-19 & GTEx\_Tissue\_Sample\_Gene\_Expression\_Profiles\_up \\ \hline
		blood\_female\_60-69\_years & 2E-19 & GTEx\_Tissue\_Sample\_Gene\_Expression\_Profiles\_up \\ \hline
		blood\_male\_30-39\_years & 3E-19 & GTEx\_Tissue\_Sample\_Gene\_Expression\_Profiles\_up \\ \hline
		blood\_female\_50-59\_years & 5E-19 & GTEx\_Tissue\_Sample\_Gene\_Expression\_Profiles\_up \\ \hline
		blood\_female\_60-69\_years & 5E-19 & GTEx\_Tissue\_Sample\_Gene\_Expression\_Profiles\_up \\ \hline
	\end{tabular}
	\label{topic/enrich/blood}
	\caption{Enrichment}
\end{table}

Enrichment test are made once for each topic, starting from the layer with more genes per 
single topic. Test are made across multiple categories.


%%tcga
\clearpage
\subsection{Run on The Cancer Genomics Atlas}
The same pipeline described so far can be applied at other datasets. In this section, the hSBM model is run on some samples from the TCGA. The principle is the same, but here samples come from cancer tissues, so there must be more complexity and variability behind the data. Moreover, being able to separate cancer samples is not always easy clinically and develop a method to do this can be fascinating and useful for the scientific community~\cite{Farver2018}. 

First of all, let's take a look at the V-measure scores. As shown in figure~\ref{fig:topic/tcga/metric} the maximum score is $\simeq 0.8$, which is quite good, comparable with the healthy GTEx scenario. The publishers of the dataset~\cite{Farver2018} obtained similar score considering just homogeneity. In this dataset, there isn't a sub-tissue label as before, but a \textit{disease type} cancer information is available. The disease type separation happens but obtain a lower score; the fact that there is no evident difference between Zipf's laws when separating data by disease type (previously shown in figure~\ref{fig:structure/tcga/fraction_of_trascriptome_disease}) means that all genes contribute to define this specific label. The hierarchic approach which separates firstly tissues and then cancer type is here necessary and useful. In fact, looking at the label \textit{disease tissue} that considers the cancer types inside each tissue the result is very encouraging. In fact the score is quite high and the results promising. To gain better scores in this situation where samples are affected by the cancer complexity and heterogeneity is probably necessary to add more genes to the network. 
\begin{figure}[htb!]
    \centering
    \includegraphics[width=0.85\linewidth]{pictures/topic/tcga/metric.pdf}
    \caption{Score across the hierarchy for TCGA. In blue the primary site labels were considered, in red the disease types and in purple the mix of the two.}
    \label{fig:topic/tcga/metric}
\end{figure}
\FloatBarrier
Looking directly into the cluster composition the tissue separation is quite good and visually appreciable. In figure~\ref{fig:topic/tcga/fraction_clustercomposition_l4_primary_site} clusters at the higher level of the hierarchy. Some tissues are well separated at this point, at the same time the model seems to group the samples by system: digestive system is the more evident.
\begin{figure}[htb!]
	\centering
	\includegraphics[width=0.9\linewidth]{pictures/topic/tcga/fraction_clustercomposition_l4_primary_site.pdf}
	\caption{Clusters of diseased tissues at the higher level of the hierarchy. Breast is well separated, such as skin and brain. Cluster 9 contains digestive systems samples from pancreas and colon.}
	\label{fig:topic/tcga/fraction_clustercomposition_l4_primary_site}
\end{figure}
Going deeper in the hierarchy the tissue separation becomes visually appreciable and all the clusters are almost tissue-specific. This is clear in figure~\ref{fig:topic/tcga/fraction_clustercomposition_l3_primary_site} which shows the primary site of the tumours well classified.
\begin{figure}[htb!]
	\centering
	\includegraphics[width=0.9\linewidth]{pictures/topic/tcga/fraction_clustercomposition_l3_primary_site.pdf}
	\caption{Normalized cluster composition from diseased samples. The primary site is here reported.}
	\label{fig:topic/tcga/fraction_clustercomposition_l3_primary_site}
\end{figure}

Going further, deep in the hierarchy, the disease type associated with each site emerges. In figure~\ref{fig:topic/tcga/fraction_clustercomposition_l2_disease_tissue} it is shown that each tissue is then separated between different disease types. Note that the pure disease type classification is not useful since certain types of tumours can appear in different sites. In this case, the power of a hierarchic approach is evident: firstly the sites are retrieved, but going further also the disease type is classified quite well.
\begin{figure}[htb!]
	\centering
	\includegraphics[width=0.9\linewidth]{pictures/topic/tcga/fraction_clustercomposition_l2_disease_tissue.pdf}
	\caption{Normalized cluster composition of diseased tissue or couple site and disease type.}
	\label{fig:topic/tcga/fraction_clustercomposition_l2_disease_tissue}
\end{figure}
\FloatBarrier
At this point when the model is demonstrated to work on healthy and diseased samples, it can be interesting to study merged healthy and diseased labels and examine how the model behaves when healthy and cancer samples are merged. It can be very useful to determine when the model identifies a diseased sample and when it is able to classify it properly.

\clearpage
\subsection{Healthy and diseased together}
In the previous sections it was demonstrated that the model works on samples from different datasets and performs well on both healthy and diseased samples. It can be interesting to see how the model behaves when both kinds of data are presented to it.
The goal of this part of work is to identify which genes or topics identify and distinguish tissues themselves and which drive cancer and are necessary to understand the differentiation between cancer types.

For this analysis data from GTEx and TCGA were still analysed, but from a particular dataset available from~\cite{Wang2017} were authors tried to unify the normalization process from different dataset and sources~\cite{Betel2018}. This is, in practice, a mixed bigger dataset; note that not every tissue is present in both GTEx and TCGA, so only common tissues are considered here. The first label considered at this point is the tissue primary site, forgetting about its status (healthy or diseased), the secondary label refers to the tissues but separates their status. For example, a healthy brain sample from GTEx and a cancer brain from TCGA share the \textit{brain} primary site label but have different secondary site assignments.

Once the model is run, the first element to look at is the V-measure; in figure~\ref{fig:topic/merged/metric_scores_primarysite} the result for the primary site is quite satisfying: clusters are very homogeneous and V-measure's peak is near $0.8$.
\begin{figure}[htb!]
    \centering
    \includegraphics[width=0.8\linewidth]{pictures/topic/merged/metric_scores_primarysite.pdf}
    \caption{V-measure score for the run with merged healthy and diseased samples. Homogeneity, completeness and mutual information are represented.}
    \label{fig:topic/merged/metric_scores_primarysite}
\end{figure}

Estimating the score also for the secondary site, or rather for the tissues with the health state and just the healthy/disease label lead to figure~\ref{fig:topic/merged/metric_scores}. This result is quite interesting, first of all even the secondary label is well classified and this happens at deeper level with respect to the one where tissues are separated; this means that firstly samples are separated by tissues then by their health state. This is very interesting because it is an evidence that the model recognize tissues, never mind where they come from, moreover the difference between datasets are not important here and so the normalization made by~\cite{Betel2018} brings no problems at this level. Moreover, looking just at the health status label the score is quite low (below $0.2$) so the model does not take over the difference between datasets.
\begin{figure}[htb!]
    \centering
    \includegraphics[width=0.8\linewidth]{pictures/topic/merged/metric_scores.pdf}
    \caption{V-measure score for the run with merged healthy and diseased samples. Primary site (brain, blood, pancreas\ldots) labels are compared with secondary labels (healthy brain, brain cancer, healthy blood, blood cancer, healthy pancreas, pancreas cancer\ldots). The health status label (healthy / score diseased) is plotted.}
    \label{fig:topic/merged/metric_scores}
\end{figure}
To conclude the score analysis a mixed score is considered (the homogeneity of the primary site is considered with the completeness of secondary label) so that the score increases if going deeper in the hierarchy the separation of a homogeneous cluster brings to the separation of the refined labels. In figure~\ref{fig:topic/merged/metric_scores_all} this score is compared with the ones obtained with LDA, hierarchical clustering and the null model. What happened here is that hierarchical Stochastic Block Model performs the best, LDA approach is good, hierarchical clustering has a quite bad score and all are better than the reshuffling null model.
\begin{figure}[htb!]
    \centering
    \includegraphics[width=0.8\linewidth]{pictures/topic/merged/metric_scores_all.pdf}
    \caption{V-measure score for run with merged healthy and diseased samples. LDA, hierarchical clustering and null model for comparison.}
    \label{fig:topic/merged/metric_scores_all}
\end{figure}

\FloatBarrier
\paragraph{Gene sets} analysis is then performed. Considering the P-value of the term which P-value was the lowest, one P-value for each topic at the level of the hierarchy where the V-measure was maximized, it is possible to realize the $-Log_{10}(\mathrm{P-value})$ histogram. The tests are quite interesting, in fact there is an enrichment with a P-value lower than $0.05$ in most cases, so it is possible to assert that topics carry some interesting information more than expected by picking up genes at random.
\begin{figure}[htb!]
    \centering
    \includegraphics[width=0.65\linewidth]{pictures/topic/merged/pvaluescrosstopic.png}
    \caption{$-Log_{10}(\mathrm{P-value})$ of the term with the lowest P-value in each topic. In orange the standard $0.05$ threshold.}
    \label{fig:topic/merged/pvaluescrosstopic}
\end{figure}
In figure~\ref{fig:topic/merged/pvaluecategories} the categories of the terms with lower P-values are shown. This explains what aspect of the samples a topic describes. The majority of terms found in topics comes from the GTEx annotation for tissue expression, many are from GO biological process, GO molecular function and some from Human phenotype ontology. Nevertheless, some topics present enrichment for \textit{NCI-60 Cancer Cell Lines}, meaning that these topics contains genes that are somehow cancer-related.
\begin{figure}[htb!]
    \centering
    \includegraphics[width=0.8\linewidth]{pictures/topic/merged/pvaluecategories.pdf}
    \caption{Categories of the terms with lower P-values in each topic.}
    \label{fig:topic/merged/pvaluecategories}
\end{figure}

Going forward in the analysis it is possible to perform enrichment test with other tools such as DAVID~\cite{huang2008bioinformatics,huang2009systematic}. Results are similar to the ones retrieved before. Tissue-related terms are found also using this tool; this confirms the absence of tool, sets or categories related biases. In figures~\ref{fig:topic/merged/DAVID_lung},~\ref{fig:topic/merged/DAVID_brain} and~\ref{fig:topic/merged/DAVID_stomach} the result from DAVID enrichment analysis. Finally, it is interesting to notice that topics are quite small (order $\simeq20$ genes), so there aren't biases that can appear doing enrichment tests on big sets.
\begin{figure}[htb!]
    \centering
    \includegraphics[width=0.8\linewidth]{pictures/topic/merged/DAVID_lung.pdf}
    \caption{Enrichment test on DAVID platform reveals lung-related genes.}
    \label{fig:topic/merged/DAVID_lung}
\end{figure}
\begin{figure}[htb!]
    \centering
    \includegraphics[width=0.8\linewidth]{pictures/topic/merged/DAVID_brain.pdf}
    \caption{Enrichment test on DAVID platform reveals brain-related genes.}
    \label{fig:topic/merged/DAVID_brain}
\end{figure}
\begin{figure}[htb!]
    \centering
    \includegraphics[width=0.8\linewidth]{pictures/topic/merged/DAVID_stomach.pdf}
    \caption{Enrichment test on DAVID platform reveals stomach-related genes.}
    \label{fig:topic/merged/DAVID_stomach}
\end{figure}

\clearpage
\paragraph{The link between topics and samples} has not been investigated so far. The probability distribution of each sample over topics $P(\text{topic}| \text{sample})$ can be estimated as \[P(\text{topic}| \text{sample})=\frac{\text{\# of edges on sample from topic}}{\text{\# of edges on sample}}\] after the model is run. Moreover, an average of all samples belonging to a topic can be estimated: $P(\text{topic}| \text{tissue})=\frac{1}{\left|tissue\right|}\sum_{sample\in tissue}P(\text{topic}| \text{sample})$.

In figure~\ref{fig:topic/merged/lifeplot} $P(\text{topic}| \text{tissue})$ is plotted for the first topics. What is clear is that in all samples there is a global trend without many differences between tissues, the topic expression differences between tissues are slightly appreciable at this point. This new point of view carries a profound and very informative message: in nature every tissue needs somehow the expression of all the genes (there is a global trend) and small differences between genes' expression are fine-tuned to obtain different tissues. In other words, it is possible to describe human tissues assuming that all genes are important and that is the fine structure of their interactions which realizes the complexity observed. In the case of diseased samples, this suggests that it should be possible to discover a cancer type not looking at a few marker genes but looking at the whole expression profile of all the genes.
\begin{figure}[htb!]
	\centering
	\includegraphics[width=0.95\linewidth]{pictures/topic/merged/lifeplot.pdf}
	\caption{$P(\text{topic} | \text{tissue})$ for some topic coloured by tissue. It is evident a global trend and in some topics there are little differences between tissues.}
	\label{fig:topic/merged/lifeplot}
\end{figure}

In order to better understand these differences between topic expression in different tissues some kind of normalization inside each topic is needed. Here it was chosen to study inside each topic which tissues are most differently expressed (in average). To do so from each $P(\text{topic}| \text{tissue})$ it was subtracted the average topic expression $\text{mean}_{\text{tissue}}(\text{topic})=\avg{P(\text{topic}| \text{tissue})}_{tissue}$ and the result was divided by the standard deviation $\sigma_{\text{tissue}}(\text{topic})$. In figure~\ref{fig:topic/merged/lifeplot_normalised_level3_hd} some most characteristic topics are reported. This analysis reveals that different tissues have a different distance from average in different topics. When a tissue is distant from the average in a topic, usually that topic means something for that particular tissue. This analysis is useful to determine what is the role of each topic, moreover if a topic reveals cancer the difference between the healthy tissue and its diseased counterpart emerges as shown in the figure.
\begin{figure}[htb!]
	\centering
	\includegraphics[width=0.85\linewidth]{pictures/topic/merged/lifeplot_normalised_level3_hd.pdf}
	\caption{$\frac{\left|P(topic | tissue) - \avg{P(\text{topic}| \text{tissue})}_{tissue}\right|}{\sigma_{\text{tissue}}(\text{topic})}$ or the distance of each tissue from the average tissue expression in each topic. Some low occurrence topics are reported. Note \textit{breast} cancer and healthy \textit{thyroid} emerge.}
	\label{fig:topic/merged/lifeplot_normalised_level3_hd}
\end{figure}

The study of the relationship between topics and samples concludes the topic modelling analysis. In the next section, all results achieved will be summarized.

\section{Results}

\draft{Using as gene set ~\cite{Ardlie2015} enrichment test can be made \cite{Kuleshov2016}}

Enrichment test are made once for each topic, starting from the layer with more genes per 
single topic. Test are made across multiple categories.