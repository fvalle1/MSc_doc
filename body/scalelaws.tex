\chapter{Scale laws}\label{ch:scalelaws}
\draft{taylor \cite{Eisler2008} \cite{altmann2016statistical}}
\section{Scaling}
It can be interesting to study how a particular gene changes its behaviour in different samples. 
This will be useful if you want to decide in which tissues a gene is expressed more than average.

First of all it could be interesting to represent the variance of expression $\sigma^2_{FPKM}$ versus 
the average FPKM $<FPKM>$ across tissues.

%%CV^2

%%Poisson

%%gamma


\paragraph{$<FPKM>$ versus occurrence}
One can be interested in finding genes that are expressed often, and what is the 
average expression of them.
To manage this it is plotted the average expression $<FPKM>$ versus the number 
of samples in which that gene is expressed that is, considering the thresholds~\ref{sec:threshold}, 
$\Sigma_j\theta (FPKM_{ij}-0,1)\theta (10^5-FPKM_{ij})$

\subsection{Tissue differentiation}
Per gene type scaling

%%GSEA cita

%%gene type distribution


