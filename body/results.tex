\chapter{Results}\label{ch:results}
Finally this work demonstrates that RNA-Sequencing datasets can be analyses from a component systems point of view.
This kind of data shows typical trends famous, for example, in linguistics, moreover some interesting biological signatures were found. RNA-Seq dataset have a great core of protein coding genes that express everywhere, this is evident looking at $U$s, Heaps' law. The presence of a power law distribution in the ranked abundances, the so called Zipf's law is observed and characterize the distribution of genes expression data.

In the first part of this work a dataset (GTEx) containing samples from healthy tissues was analysed. One of the most interesting evidences was the presence of many different Zipf's law if one considers each tissue independently. Very similar results were obtained considering TCGA, a dataset containing thousands of samples of cancer tissues.

The power law distribution encouraged to explore the possibility of using a topic model approach to reveal the hidden structure of these datasets. This approach is useful both to find clusters of samples that share some properties and to find the relation between genes and samples.

Many goals were achieved during these analysis.
First of all it was developed a pipeline that begins with creating a network with useful genes and samples, this network is than processed with hierarchic stochastic block model topic model algorithm and then analysed.

One interesting result is that this method is able to reproduce the distinction between different tissues, this is evident looking at the cluster composition. Moreover if one defines more objective metrics based on the entropy the score is quite high, this encouraged further analysis.
In particular in many cases the model not only reproduce the main tissue label, but was demonstrated that running along the hierarchy of the clusters even the sub tissue specific labels were distinguished. The mutual information score confirms this behavior of the model were tissues were separated at an higher level of the hierarchy and in next one the sub tissue were explained.

A null model realized shuffling the labels confirms that the results achieved are non trivial and represent somehow the real tissue structure of the data.

The results were compared with more standard approaches such as Latent Dirichlet Allocation and hierarchical clustering. In both cases the results of the approach presented in this work are better than the standard and obtain higher scores. Moreover topic modeling (both LDA and hSBM) is better than standard algorithms. This confirms the good quality of a topic model approach.

On the other side looking at the block of genes, the so called topics, enrichment tests confirms that the topics represent interesting group of genes. In particular some dataset-specific labels were found in GTEx.

In the end the relation between samples and topics reveals that the topics the have uncommon behavior in the samples of a specific tissue enrich for a function specific for that tissue. The distribution of the topic abundance across samples reveals that it is possible, despite to what a LDA approach does, to describe the tissue differentiation with topic that varies slightly between samples. Biologically this means that all genes are necessary everywhere and a fine tuning of their expression differentiate by tissue.

In the end it was demonstrated that analyzing data coming from merged dataset the differentiation is still evident and going forward in the hierarchy depth the separation involves also the healthy or diseased status. The tissue separation in the firsts layer confirms that what the algorithm does is separating tissues and there is no, evident, bias between datasets.