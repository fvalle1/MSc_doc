\section{Results}\label{sec:topic/results}
The analysis using topic modelling leads to many interesting results.

The first result achieved is the development of a model that is able to reproduce the distinction between different tissues from RNA-sequencing datasets. This is evident looking at the cluster composition. Moreover, if one defines more objective metrics based on the entropy the score is quite high, this encouraged further analysis. In particular, in many cases, the model not only reproduces the main tissue classification but was demonstrated that at different layers of the hierarchy even the sub-tissue labels are distinguished. The mutual information score confirms this model's behaviour: tissues are separated at a higher level of the hierarchy and in the deeper layers the sub-tissues are distinguished.

A null model realized shuffling the labels confirms that the results achieved are non-trivial. Studying some quantities such as the fraction of cluster with the same label or the number of labels in a cluster and comparing these with the null models' ones it is possible to affirm that clusters are more homogeneous than expected.

The output of the model presented in this work (hSBM) was compared with more standard approaches such as Latent Dirichlet Allocation and hierarchical clustering. hSBM outputs better results with respect to standard approaches, moreover, it gains higher scores. An interesting fact is that topic modelling (both hSBM and LDA) is better than standard algorithms. This confirms the good quality of a topic model approach and, inside topic models, hSBM seems better than LDA. All algorithms are distant from the null model as highly expected.

It was also demonstrated that not only clustering of the sample was satisfying but even the genes' classification is interesting. If one looks at the block of genes, the so-called topics, enrichment tests confirm that topics represent interesting group of genes. In particular, some dataset-specific labels were found in GTEx analysis.

In the end, the relationship between samples and topics reveals interesting facts. The distribution of the topic abundance across samples reveals that it is possible to describe the tissue differentiation as a complex mechanism of relationships between genes' expression. This isn't possible using an LDA approach where genes can have either a uniform or peaked distribution. Biologically this means that all genes are necessary everywhere and a fine-tuning of their expression differentiate by tissue.

The sample clustering, topic analysis and the relationship between samples and topics were made on three different datasets. GTEx containing just healthy samples, TCGA containing cancer samples and~\cite{Wang2017} which merged the two. Analysing the dataset with both healthy and diseased samples the differentiation between tissues is still evident and going deeper in the hierarchy the separation involves also the healthy or diseased status. The tissue separation in the firsts layer confirms that what the algorithm does is separating tissues and there is no, evident, bias between datasets.

Each of these cases reveals interesting facts. Being able to reproduce GTEx labels is a good benchmark of the quality of the algorithm; to accurately reproduce TCGA labels is the real challenge that can improve scientific community knowledge and here was partially achieved. Analysing both at the same time helps in understanding which genes are somehow involved in cancer development.