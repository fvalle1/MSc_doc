\chapter{Conclusions}\label{ch:conclusions}
Finally this work demonstrates that RNA-Sequencing datasets can be analyses from a component systems point of view.
This kind of data shows typical trends famous, for example, in linguistics, moreover some interesting biological signatures were found. RNA-Seq dataset have a great core of protein coding genes that express everywhere, this is evident looking at $U$s, Heaps' law. The presence of a power law distribution in the ranked abundances, the so called Zipf's law is observed and characterize the distribution of genes expression data.

In the first part of this work a dataset (GTEx) containing samples from healthy tissues was analysed. One of the most interesting evidences was the presence of many different Zipf's law if one considers each tissue independently. Very similar results were obtained considering TCGA, a dataset containing thousands of samples of cancer tissues.

The power law distribution encouraged to explore the possibility of using a topic model approach to reveal the hidden structure of these datasets. This approach is useful both to find clusters of samples that share some properties and to find the relation between genes and samples.

Many goals were achieved during these analysis.
First of all it was developed a pipeline that begins with creating a network with useful genes and samples, this network is than processed with hierarchic stochastic block model topic model algorithm and then analysed.


In conclusion topic model reveal itself as a useful approach to this kind of data.

Verified that benchmark give good results the next goal would be to 

\draft{future: Loredana, Jacopo, topic on sampling, Jonathan}